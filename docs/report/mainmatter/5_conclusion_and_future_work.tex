\chapter{Conclusion and Further Work}
\label{sec:conclusion}

In this work, we have proposed the projection step of the cross-lingual named entity recognition
pipeline as an integer linear programming (ILP) problem \eqref{eq:ilp}.
This formulation aims to extract candidates within the target sentence and match them
with the source entity provided from the preceding step of the XLNER pipeline. The likelihood
of such matches is represented by the matching scores, which should be maximized under
the natural constraints that the source entity cannot be projected onto overlapping candidates,
and exists a limit on the number of projections for each source entity.

Three distinct options for matching costs have been proposed, each offering different relevant
connections to the projection problem, including alignment-based, NER-model-based, and translation-based scores.
A notable feature that differs our formulation from others is its capacity to use matching costs
computed based on different principle and motivations and fuse them together to conceal the weaknesses
inherent in each individual score.

For candidate extraction, a straightforward approach has been proposed, which considers all word
n-grams constrained by the maximum length of the target sentence as potential candidates for matching.
Furthermore, it has been demonstrated that, for certain scores, specifically alignment-based, this
set can be reduced without compromising any optimal solution regarding the constraints of type
\eqref{eq:num_proj_const} of form "less" or "less or equal", thus enabling a more efficient computation
of the ILP solution.

The complexity of the proposed ILP problem remains an open question. Nonetheless, it has been
shown that, in the absence of constraints limiting the number of projections for each source
entity, the problem can be solved by a polynomial-time algorithm. This property arises because the
non-overlapping constraints, by construction, induce an interval graph. Thus, even if the problem is NP-hard,
then the constraints \eqref{eq:num_proj_const} play a critical role in its complexity.

Additionally, a fast greedy approximate algorithm has been developed for practical application,
yielding solutions that satisfy non-overlapping constraints and constraints of
type \eqref{eq:num_proj_const} when they are of form "less" or "less or equal". However, this algorithm continues
to demonstrate superior performance from the application standpoint even when constraints \eqref{eq:num_proj_const} have other forms, as the
enforcement of certain constraints typically leads to less optimal solutions from an application
perspective.

It is also important to highlight that the constraints restricting the number of projected candidates
for each source entity, particularly with respect to constraints of type "equal", "greater", "greater or equal", may result
in the optimal solution not aligning with the optimal labeling of the target sentence, as the
exact solver has to enforce these constraints and therefore may break optimal projection from the application
point of view.

We have evaluated our approach in isolation (focusing solely on the projection step) utilizing the
Europarl-NER dataset, as well as within the complete XLNER pipeline using the MasakhaNER2 dataset.
The evaluation results indicate that the highest metrics for the application problem are attained when
the constraints of type \eqref{eq:num_proj_const} are defined as \( \leq 1 \). This can be attributed
to the expectation of a one-to-one correspondence between source and target entities for languages
sharing similar writing systems. However, due to potential errors introduced by the source NER
model on the previous step of the pipeline—such as incorrect predictions of non-existent entities—and
the imperfect nature of matching costs derived from various models, the solver must be permitted to
not match all source entities.

Experiments demonstrate that the proposed projection method consistently outperforms
heuristic word-to-word alignment-based and model transfer pipelines, particularly in scenarios
involving fused scores that incorporate all types of scores, which represents a principal advantage
of our method. While heuristics can occasionally outperform the ILP-based projection step utilizing
alignment-based matching scores due to their aggressive filling of gaps in alignments, the proposed
NER model-based score frequently exceeds the performance of model transfer that utilize the
same NER model outputs, achieving improvements of up to 1.5 times. This superior performance can be
attributed to the projection step's ability to filter out predicted target entities that do not
correspond to any source entity.

The findings presented here advocate for further research directions.
First, it is essential to determine the complexity of the proposed problem, either through
reduction to or from a problem with known complexity or by providing a polynomial-time algorithm to solve this problem.
Secondly, novel matching costs could be developed to extend and enhance existing ones. Particularly
intriguing are matching scores that can take negative values, as this flexibility may enable the
application of alternative constraints of type \eqref{eq:num_proj_const}, such as "greater or equal".
An additional direction for research involves the refinement of candidate extraction strategies,
as the current set of all n-grams up to a defined maximum length is excessively large, resulting
in slower computations for both the solution and matching scores. In addition to the options
discussed in the preceding sections, such as employing large language models (LLMs) or NER models
to predict candidates, utilizing part-of-speech tags should also be considered, as NER entities
typically correspond to noun phrases. Furthermore, it makes sense to evaluate the proposed method in
settings where the source and target languages have completely different writing systems, for example
for Chinese, Thai, and Japanese.

From a broader perspective, the proposed ILP problem for the projection step of the XLNER pipeline
can be generalized to other Natural Language Processing tasks. For example, in the word-to-word
alignment problem, every word consists of tokens, and it is necessary to match words from the source
language with words from the target language based on their tokens. In this context, the set of target
candidates is equivalent to the set of tuples of tokens corresponding to all words in the target
sentence; therefore, there is no overlap among different candidates. The remaining challenge
is to derive a method for computing matching scores. One approach can involve adopting the methodology
utilized by current neural-based aligners and calculating these scores based on cosine similarities
between embeddings for each token.

In summary, our proposed formulation of the projection step as an ILP problem generalizes numerous
previous methods and enables their integration, addressing the drawbacks of individual methods.
By analyzing the solutions to the problem in conjunction with the corresponding matching scores,
we gain insight into why particular entities are projected onto specific target candidates, thus
enhancing interpretability, especially when compared to model transfer or the heuristic word-to-word
alignment algorithm. Nevertheless, interpretability remains limited due to the fact that all
proposed matching scores are computed utilizing neural network-based models.
