\chapter{Methodology}
\label{sec:methodology}
The overall idea of formalization the projection step of the XLNER pipeline
is to represent it as a matching the source entities and ranges of words
of the target sentence, called candidates. It is shown on the figure \ref{fig:cand_matching}.
The set of source entities that we need to project is given, so let's take some set of
continuous word ranges from the original sentence in the target language
and consider them as a possible projection of some entities. Then compute
quantities how likely a source entity should be projected onto every candidate.
And then we should take such a combination of source entities-target candidates
that maximize overall sum of selected tuples. But we need to take in mind
natural constraints that should be ensured. Assume two source entities
projects onto overlapping candidates, then we cannot precisely answer which source
entity overlapped words are projected to, therefore we should prohibit projection
onto overlapped candidates. Also, it makes sense to limit how many candidates every
source entity can be projected to, since usually we expect 1-to-1 correspondence
between source entities and entities of the target sentence. Based on this simple
properties let's build an integer linear optimization problem and study it.

\begin{figure*}[h]
    \centering
    \begin{tikzpicture}[node distance=-0.1,
            every node/.style={text centered,
                    text height=2ex,
                    text depth=.25ex,
                },
            loc/.style={fill=orange!30, rounded rectangle, label={[anchor=center,font=\tiny\bfseries\sffamily]above:#1-LOC}},
            per/.style={fill=green!30, rounded rectangle, label={[anchor=center,font=\tiny\bfseries\sffamily]above:#1-PER}},
            cand/.style={fill=blue!30, rounded rectangle},]

        \node[per={B}, rounded rectangle east arc=none](George_src){George};
        \node[per={I}, rounded rectangle west arc=none, right=of George_src](Washington_src){Washington};
        \node[right=of Washington_src](is_src){is};
        \node[right=of is_src](the_src){the};
        \node[right=of the_src](first_src){first};
        \node[right=of first_src](president_src){president};
        \node[right=of president_src](of_src){of};
        \node[right=of of_src](the_src){the};
        \node[loc={B}, rounded rectangle east arc=none, right=of the_src](United_src){United};
        \node[loc={I}, rounded rectangle west arc=none, right=of United_src](States_src){States};

        \node[cand, rounded rectangle east arc=none, below=of George_src, yshift=-1.5cm](George_tgt){George};
        \node[cand, rounded rectangle west arc=none, right=of George_tgt](Washington_tgt){Washington};
        \node[right=of Washington_tgt](ist_tgt){ist};
        \node[right=of ist_tgt](der_tgt){die};
        \node[right=of der_tgt](erste_tgt){erste};
        \node[right=of erste_tgt](president_tgt){Präsident};
        \node[right=of president_tgt](der_tgt){der};
        \node[cand, rounded rectangle east arc=none, right=of der_tgt](Vereinigten_tgt){Vereinigten};
        \node[cand, rounded rectangle west arc=none, right=of Vereinigten_tgt](Staaten_tgt){Staaten};

        \node[text=gray, font=\scriptsize, above=of first_src, yshift=0.2cm, xshift=0.2cm](source){Source labeled sentence};
        \node[text=gray, font=\scriptsize, below=of source, yshift=-3cm]{Original sentence with extracted candidates};

        \draw[->] (George_src.south east) -- node[left]{\(c_{11}\)} (George_tgt.north east);
        \draw[->] (United_src.south east) -- node[right]{\(c_{22}\)} (Vereinigten_tgt.north east);
        \draw[->] (George_src.south east) -- node[above left, yshift=0.1cm, xshift=-0.2cm]{\(c_{12}\)} (Vereinigten_tgt.north east);
        \draw[->] (United_src.south east) -- node[below left]{\(c_{21}\)} (George_tgt.north east);
    \end{tikzpicture}
    \caption{Illustration of the proposed idea of matching source entities and candidates in the target sentence}
    \label{fig:cand_matching}
\end{figure*}

\section{Formulation of the ILP problem}
First of all let's start with a definition of overlapping.

\begin{definition}[Relation of overlapping] \label{def:overlapping}
    We consider that target candidates \linebreak
    \( \tgt{p_1} = (i_{\tgt{p_1}}, j_{\tgt{p_1}}), \tgt{p_2} = (i_{\tgt{p_2}}, j_{\tgt{p_2}}) \in T \) overlap if and only if
    \( i_{\tgt{p_1}} \leq j_{\tgt{2}} \land i_{\tgt{p_2}} \leq j_{\tgt{1}} \), i. e. sets of word indices are
    not disjoint. We will denote that candidates overlap using
    the following notation: \( \tgt{p_1} \cap \tgt{p_2} \neq \emptyset \).
\end{definition}

Having defined all necessary objects we can finally formulate the ILP problem for a
projection step of the XLNER pipeline.

Let \( S \) be a set of source entities, \( T \) -- set of target candidates and \( \cap \subset T^2 \) is
a relation of overlapping, then the projection ILP problem is the following:
\begin{align}
    \label{eq:objective}
     & \max\limits_x \sum\limits_{(\src{p}, \tgt{p}) \in S \times T} c_{\src{p}, \tgt{p}} x_{\src{p}, \tgt{p}}                                                                            \\
     & \text{subject to} \notag                                                                                                                                                           \\
    \label{eq:num_proj_const}
     & \sum\limits_{\tgt{p} \in T} x_{\src{p}, \tgt{p}} \lessgtr n_{proj}                                      & \forall \src{p} \in S                                                    \\
    \label{eq:non_overlap_const}
     & x_{\src{p_1}, \tgt{p_1}} + x_{\src{p_2}, \tgt{p_2}} \leq 1                                              & \forall (\src{p_1}, \src{p_2}, \tgt{p_1}, \tgt{p_2}) \in \hat{\Pi}(S, T) \\
    \label{eq:binary_prog}
     & x_{\src{p}, \tgt{p}} \in \{ 0, 1 \}                                                                     & \forall (\src{p}, \tgt{p}) \in S \times T
\end{align}
where \( \hat{\Pi}(S, T) \) is a set of source entities -- target candidates combinations that can not be
projected onto together since they are overlapping. This set is defined as follows:
\begin{equation}  \label{eq:overlapping_set}
    \begin{aligned}
        \hat{\Pi}(S, T) = \Big\{ (\src{p_1}, \src{p_2}, \tgt{p_1}, \tgt{p_2}) \Big| \src{p_1}, \src{p_2} \in S, \tgt{p_1}, \tgt{p_2} \in T, \quad \tgt{p_1} \cap \tgt{p_2} \neq \emptyset, \\
        (\src{p_1} \neq \src{p_2}) \lor (\tgt{p_1} \neq \tgt{p_2}) \Big\}
    \end{aligned}
\end{equation}

Here every variable \( x_{\src{p}, \tgt{p}} \) represents whether the source entity \( \src{p} \in S \)
is projected to the target candidates \( \tgt{p} \in T \).
The set of constraints \eqref{eq:non_overlap_const} ensures that it is impossible to have a situation when
two or one source entity are being projected onto overlapping target candidates.
And other set of constraints \eqref{eq:num_proj_const} limits number of projections
for every source entity. Note, that the inequality sign is not fixed since any form
could make sense: less, less or equal and equal can be used to ensure that we don't
project a source entity to as much available candidates as possible as long as it
increase the objection value, and great or great or equal can be a way to enforce
that a source entiry will be projected at least specified number of times, e.g. once,
and solver will not simply ignore it. It is clear that \( n_{proj} = 0 \) is a corner
case and usually doesn't make sense except greater inequality when it simply
eliminates any limits and equivalent to not have any constraints of such type.

Also it is worth to note that definition of the ILP problem is not bounded to a specific form of sets \( S, T \)
and overlapping relation \( \cap \subset T^2 \), so given any sets and overlapping relation with the same properties
as for \ref{def:overlapping}, the ILP problem won't change since it doesn't use explicitly word indices.

However the number of constraints \eqref{eq:non_overlap_const} scales as a cube with respect to
length of the target sentence, assuming that we use all possible continuous subranges as candidates, and as
a square with respect to the number of source entities.
It can make this problem harder to solve. Therefore we need to try to lower number of constraints
of this type. For this let's take a look at the overlapping relation that a basis of it.
\begin{lemma} \label{lemma:not_transitive}
    The relation of overlapping is not transitive.
\end{lemma}
\begin{proof}
    We will proof this by showing a counterexample.
    Assume the following target candidates
    \[
        a = (2, 4) \quad b = (1, 2) \quad c = (4, 5)
    \]
    By the definition of the overlapping relation \( a \cap b \neq \emptyset \) and
    \( a \cap c \neq \emptyset \) but \( b \) and \( c \) are not overlapping.
\end{proof}

This fact leads to the impossibility to split all target candidates into groups of mutually
overlapped candidates and choice at maximum one candidate from every group to match with a source entity.
\begin{corollary}
    It is impossible to partition set of target candidates such that every partition consists of
    candidates that are pairwise overlapping with each other and there are no overlapping candidates
    that are in different partitions.
\end{corollary}
\begin{proof}
    By Lemma \ref{lemma:not_transitive} the overlapping relation is not transitive and therefore
    is not an equivalence relation that implies we can not construct a quotient set.
\end{proof}

Nevertheless it is possible to make number of constraints \eqref{eq:non_overlap_const} lower.
The idea behind the reduction is that if it is impossible to project one or any two source entities to
overlapped candidates it means that it impossible to project any number of source entities to these candidates.
Thus we can just sum up constraints over all source entities.
We can do it because all variables \( x_{\src{p}, \tgt{p}} \) are binary and therefore
can not be negative.
\begin{theorem}
    Set of constraints \eqref{eq:non_overlap_const} is satisfied if and only if
    the following sets of reduced constraints are satisfied:
    \begin{align*}
         & \sum\limits_{\src{p} \in S} (x_{\src{p}, \tgt{p_1}} + x_{\src{p}, \tgt{p_2}}) \leq 1                                & \forall (\tgt{p_1}, \tgt{p_2}) \in \Pi(T) \\
         & \sum\limits_{\src{p} \in S} x_{\src{p}, \tgt{p}} \leq 1
         & \forall \tgt{p} \in T \Big| \nexists \tgt{p_2} \in T: \tgt{p} \neq \tgt{p_2}, \tgt{p} \cap \tgt{p_2} \neq \emptyset                                             \\
    \end{align*}
    where
    \begin{equation*}
        \Pi(T) = \left\{ (\tgt{p_1}, \tgt{p_2}) \Big| \tgt{p_1}, \tgt{p_2} \in T,
        \quad \tgt{p_1} \cap \tgt{p_2} \neq \emptyset,
        \tgt{p_1} \neq \tgt{p_2} \right\}
    \end{equation*}
\end{theorem}
\begin{proof}
    \textit{Necessity:} Assume that constraints \eqref{eq:non_overlap_const} are satisfied, but
    the proposed constraints are not. Then there are two options:
    \[
        \exists (\tgt{p_1}, \tgt{p_2}) \in \Pi(T) \Bigg| \quad
        \sum\limits_{\src{p} \in S} (x_{\src{p}, \tgt{p_1}} + x_{\src{p}, \tgt{p_2}}) > 1
    \]
    or
    \[
        \exists \tgt{p} \in T \Big|
        \nexists \tgt{p_2} \in T: \tgt{p} \neq \tgt{p_2}, \tgt{p} \cap \tgt{p_2} \neq \emptyset
        \Bigg| \quad \sum\limits_{\src{p} \in S} x_{\src{p}, \tgt{p}} > 1
    \]

    Consider the first option:
    \begin{align*}
         & \sum\limits_{\src{p} \in S} (x_{\src{p}, \tgt{p_1}} + x_{\src{p}, \tgt{p_2}}) > 1
        \stackrel{\eqref{eq:binary_prog}}{\implies}                                                                     \\
         & \exists \src{p_1}, \src{p_2} \in S \Big| x_{\src{p_1}, \tgt{p_1}} = 1, x_{\src{p_2}, \tgt{p_2}} = 1 \implies \\
         & x_{\src{p_1}, \tgt{p_1}} + x_{\src{p_2}, \tgt{p_2}} = 2 > 1
    \end{align*}
    that contradicts our assumption that constraints \eqref{eq:non_overlap_const} are satisfied.

    The same result we can obtain for the second case:
    \begin{align*}
         & \sum\limits_{\src{p} \in S} x_{\src{p}, \tgt{p}} > 1
        \stackrel{\eqref{eq:binary_prog}}{\implies}                                                                                           \\
         & \exists \src{p_1}, \src{p_2} \in S, \src{p_1} \neq \src{p_2} \Big| x_{\src{p_1}, \tgt{p}} = 1, x_{\src{p_2}, \tgt{p}} = 1 \implies \\
         & x_{\src{p_1}, \tgt{p}} + x_{\src{p_2}, \tgt{p}} = 2 > 1
    \end{align*}

    Thus we proved necessity.

    \textit{Sufficiency:}
    Suppose that the following constraints are satisfied:
    \[
        \sum\limits_{\src{p} \in S} (x_{\src{p}, \tgt{p_1}} + x_{\src{p}, \tgt{p_2}}) \leq 1
        \qquad \forall (\tgt{p_1}, \tgt{p_2}) \in \Pi(T)
    \]
    Then the sum could be equal to \( 0 \) or \( 1 \). If it is zero then none entities are projected onto
    any of these overlapped candidates and therefore constraints \eqref{eq:non_overlap_const} are not violated.
    Let's look at case when the sum is equal to \( 1 \).
    \begin{equation*} \label{eq:derivation_of_const_reduction}
        \begin{aligned}
             & \sum\limits_{\src{p} \in S} (x_{\src{p}, \tgt{p_1}} + x_{\src{p}, \tgt{p_2}}) = 1
            \stackrel{\eqref{eq:binary_prog}}{\implies}                                                       \\
             & \exists! \src{p} \in S, \exists! \tgt{p} \in \{ \tgt{p_1}, \tgt{p_2}\}
            \Bigg| x_{\src{p}, \tgt{p}} = 1 \implies                                                          \\
             & \forall \src{p_1}, \src{p_2} \in S, x_{\src{p_1}, \tgt{p_1}} + x_{\src{p_2}, \tgt{p_2}} \leq 1
        \end{aligned}
    \end{equation*}
    Therefore we proved that
    \[
        x_{\src{p_1}, \tgt{p_1}} + x_{\src{p_2}, \tgt{p_2}} \leq 1 \qquad
        \forall (\src{p_1}, \src{p_2}, \tgt{p_1}, \tgt{p_2}) \in \hat{\Pi'}(S, T)
    \]
    where
    \begin{align*}
        \hat{\Pi'}(S, T) = \Big\{ (\src{p_1}, \src{p_2}, \tgt{p_1}, \tgt{p_2}) \Big| \src{p_1}, \src{p_2} \in S, \tgt{p_1}, \tgt{p_2} \in T, \quad \tgt{p_1} \cap \tgt{p_2} \neq \emptyset, \\
        \tgt{p_1} \neq \tgt{p_2} \Big\}
    \end{align*}

    But it is not the same as set \eqref{eq:overlapping_set}:
    \[
        \hat{\Pi}(S, T) \setminus \hat{\Pi'}(S, T) = \Big\{ (\src{p_1}, \src{p_2}, \tgt{p_1}, \tgt{p_2}) \Big| \src{p_1}, \src{p_2} \in S, \tgt{p_1}, \tgt{p_2} \in T, \tgt{p_1} = \tgt{p_2},
        \src{p_1} \neq \src{p_2} \Big\}
    \]
    So we need to check whether constraints \eqref{eq:non_overlap_const} are satisfied on this set difference.

    First of all, let's notice that
    \begin{multline*}
        \forall (\tgt{p_1}, \tgt{p_2}) \in \Pi(T)
        \qquad
        \sum\limits_{\src{p} \in S} (x_{\src{p}, \tgt{p_1}} + x_{\src{p}, \tgt{p_2}})
        =
        \sum\limits_{\src{p} \in S} x_{\src{p}, \tgt{p_1}} +
        \sum\limits_{\src{p} \in S} x_{\src{p}, \tgt{p_2}} \leq 1
        \implies                        \\
        \forall \tgt{p} \in T \Bigg| \exists \tgt{p_2} \in T: \tgt{p} \neq \tgt{p_2}, \tgt{p} \cap \tgt{p_2} \neq \emptyset
        \quad
        \sum\limits_{\src{p} \in S} x_{\src{p}, \tgt{p}} \leq 1
    \end{multline*}
    Therefore there exists at most one source entity \( \src{p} \in S \) that are projected
    to the target candidate \( \tgt{p} \).
    And using exactly the same derivation as above we get:
    \begin{multline*}
        \forall \tgt{p} \in T \Bigg| \exists \tgt{p_2} \in T: \tgt{p} \neq \tgt{p_2}, \tgt{p} \cap \tgt{p_2} \neq \emptyset
        \quad
        \sum\limits_{\src{p} \in S} x_{\src{p}, \tgt{p}} \leq 1
        \stackrel{\eqref{eq:binary_prog}}{\implies}                                                                                                                                                       \\
        \begin{aligned}
             & \sum\limits_{\src{p} \in S} x_{\src{p}, \tgt{p}} = 0
            \stackrel{\eqref{eq:binary_prog}}{\implies}
            \forall \src{p} \in S, x_{\src{p}, \tgt{p}} = 0         \\
            \text{or}                                               \\
             & \sum\limits_{\src{p} \in S} x_{\src{p}, \tgt{p}} = 1
            \stackrel{\eqref{eq:binary_prog}}{\implies}
            \exists! \src{p} \in S, x_{\src{p}, \tgt{p}} = 1        \\
        \end{aligned} \implies \\
        x_{\src{p_1}, \tgt{p}} + x_{\src{p_2}, \tgt{p}} \leq 1 \qquad
        \begin{aligned}
             & \forall \src{p_1}, \src{p_1} \in S, \src{p_1} \neq \src{p_2}                                                        \\
             & \forall \tgt{p} \in T \Bigg| \exists \tgt{p_2} \in T: \tgt{p} \neq \tgt{p_2}, \tgt{p} \cap \tgt{p_2} \neq \emptyset
        \end{aligned}
    \end{multline*}
    So, only the case, when entity has no overlapped entity that are
    not equal to it, remains. But this case is fully covered by the second part of the proposed reduced constraints:
    \[
        \sum\limits_{\src{p} \in S} x_{\src{p}, \tgt{p}} \leq 1 \qquad
        \forall \tgt{p} \in T \Big| \nexists \tgt{p_2} \in T: \tgt{p} \neq \tgt{p_2}, \tgt{p} \cap \tgt{p_2} \neq \emptyset
    \]
    And by considering that this sum can be either \( 0 \) or \( 1 \) and making
    exactly the same derivation we conclude that it implies that
    constraints \eqref{eq:non_overlap_const} are satisfied.
\end{proof}

The theorem allows us to reduce number of constraints and make them scale as a constant
with respect to the number of source entities. As a result we end up with the following final formulation of the ILP problem for the projection step
of the XLNER pipeline.
\begin{equation} \label{eq:ilp}
    \begin{aligned}
         & \max\limits_x \sum\limits_{(\src{p}, \tgt{p}) \in S \times T} c_{\src{p}, \tgt{p}} x_{\src{p}, \tgt{p}}                                                                                                                       \\
         & \text{subject to}                                                                                                                                                                                                             \\
         & \sum\limits_{\tgt{p} \in T} x_{\src{p}, \tgt{p}} \lessgtr n_{proj}                                      & \forall \src{p} \in S                                                                                               \\
         & \sum\limits_{\src{p} \in S} (x_{\src{p}, \tgt{p_1}} + x_{\src{p}, \tgt{p_2}}) \leq 1                    & \forall (\tgt{p_1}, \tgt{p_2}) \in \Pi(T)                                                                           \\
         & \sum\limits_{\src{p} \in S} x_{\src{p}, \tgt{p}} \leq 1                                                 & \forall \tgt{p} \in T \Big| \nexists \tgt{p_2} \in T: \tgt{p} \neq \tgt{p_2}, \tgt{p} \cap \tgt{p_2} \neq \emptyset \\
         & x_{\src{p}, \tgt{p}} \in \{ 0, 1 \}                                                                     & \forall (\src{p}, \tgt{p}) \in S \times T
    \end{aligned}
\end{equation}

\section{Candidates extraction}
n-grams
\eqref{eq:ilp}
T5, LLM, high recall NER

\section{Matching score}
\subsection{Alignment-based}

\begin{equation} \label{eq:align_cost}
    c_{\src{p}, \tgt{p}}
\end{equation}

shrink candidate set + prove

\begin{lemma}
    If append unaligned word to a candidate cost will be smaller
\end{lemma}
\begin{proof}
    Trust me!
\end{proof}

\begin{theorem}
    Can remove candidates outside left and right most aligned words
\end{theorem}
\begin{proof}
    Trust me!
\end{proof}

\subsection{NER model-based}

model transfer for cand eval

\begin{equation} \label{eq:ner_cost}
    c_{\src{p}, \tgt{p}} =
\end{equation}

\subsection{Translation score-based}

nmtscore
\begin{equation} \label{eq:nmt_cost}
    c_{\src{p}, \tgt{p}}
\end{equation}

\subsection{Fused score}

ner score without classes (max over all)

weighted sum of scores
\begin{equation} \label{eq:fused_cost}
    c_{\src{p}, \tgt{p}}
\end{equation}

\section{Analysis of the ILP problem}
Despite the fact that the proposed ILP problem is just a particular case
of well-studied the general binary programming problem it doesn't imply that
it inherits all properties of it. For example, maximum bipartite matching problem
can be formulated as an instance of the ILP problem, but still can be solved in a
polynomial time, whereas general case of ILP is NP-hard. That motivates a deeper analysis
of the proposed formulation.

\subsection{Complexity}
Constraints, especially \eqref{eq:non_overlap_const}, play a crucial role in a complexity
of the proposed problem. In the case if all candidates are nor overlapping and constraints
\eqref{eq:num_proj_const} has a form of equalities with \( n_{proj} = 1 \) the problem reduces to
an instance of the weighted bipartite matching problem, that is in complexity class \( P \). But let's study
more general case.

For the sake of convenience we will use the first form \eqref{eq:objective}--\eqref{eq:binary_prog}
of the ILP problem in this section, but since it equivalent to the \eqref{eq:ilp} it doesn't affect
the results of the analysis.

The usual way to prove the complexity of some problem is to reduce other problem with known complexity to
an instance of the studied one. Thus let's consider the maximum independent set problem \cite{pemmaraju2003computational}.

\begin{figure}[ht]
    \begin{subfigure}{.5\textwidth}
        \centering
        \begin{tikzpicture}[every node/.style = {draw, circle}]
            \node[fill=green!10] (1) {1};
            \node[right=of 1] (2) {2};
            \node[below right=of 2, fill=green!10] (3) {3};
            \node[below=of 1, fill=green!10] (4) {4};
            \node[left=of 4] (5) {5};
            \node[below right=of 4] (6) {6};

            \graph{
                (1) -- (2) -- (4) -- (6) -- (5) -- (4),
                (6) -- (2) -- (3)
            };
        \end{tikzpicture}
        \caption{Maximum independent set problem (MaxIS). Vertices that form an optimal solution are colored in \textbf{\textcolor{green!50}{green}}}
    \end{subfigure}
    \begin{subfigure}{.5\textwidth}
        \centering
        \begin{tikzpicture}[
                node distance=0.1,
                realstate/.style = {draw, circle, font=\scriptsize},
                dummystate/.style = {draw, circle, minimum width=4, inner sep=6, fill=blue!10}
            ]
            \node[dummystate] (src_1) {};
            \node[right=of src_1, xshift=10, dummystate] (src_2) {};
            \node[right=of src_2, xshift=10, dummystate] (src_3) {};

            \node[below=of src_1, fill=green!10, xshift=-50, yshift=-1.75cm, realstate] (1) {1};
            \node[right=of 1, realstate] (2) {2};
            \node[right=of 2, fill=green!10, realstate] (3) {3};
            \node[right=of 3, fill=green!10, realstate] (4) {4};
            \node[right=of 4, realstate] (5) {5};
            \node[right=of 5, realstate] (6) {6};
            \node[right=of 6, dummystate](dummy_1) {};
            \node[right=of dummy_1] (dots) {\dots};
            \node[right=of dots, dummystate] (dummy_2) {};

            \node[above=of src_2, text=gray, font=\scriptsize] {Source nodes};
            \node[below=of 5, text=gray, font=\scriptsize] {Target nodes};

            \draw[green] (src_1) -- (1);
            \draw (src_1) -- node[above left, xshift=3, rotate=60, black, font=\footnotesize]{c=1} (2);
            \draw (src_1) -- (3);
            \draw (src_1) -- (4);
            \draw (src_1) -- (5);
            \draw (src_1) -- (6);
            \draw[gray!40] (src_1) -- (dummy_1);
            \draw[gray!40] (src_1) -- (dummy_2);
            \draw (src_2) -- (2);
            \draw[green] (src_2) -- (3);
            \draw (src_2) -- (4);
            \draw (src_2) -- (5);
            \draw (src_2) -- (6);
            \draw[gray!40] (src_2) -- (dummy_1);
            \draw[gray!40] (src_2) -- (dummy_2);
            \draw (src_3) -- (2);
            \draw (src_3) -- (3);
            \draw[green] (src_3) -- (4);
            \draw (src_3) -- (5);
            \draw (src_3) -- (6);
            \draw[gray!40] (src_3) -- (dummy_1);
            \draw[gray!40] (src_3) -- node[above right, black, font=\footnotesize]{c=0} (dummy_2);
        \end{tikzpicture}
        \caption{Reduction to the proposed ILP problem. Dummy nodes are colored in \textcolor{blue!60}{blue}}
    \end{subfigure}
    \caption{The diagram of the general idea of the reduction MaxIS problem to the proposed ILP problem}
    \label{fig:maxis_reduction}
\end{figure}

The maximum independent set problem is a problem of finding a subset of vertices of some undirected
graph with maximum cardinality, such that it doesn't contain any vertices connected by edges of the graph.
An example of the maximum independent set problem is depicted on the figure \ref{fig:maxis_reduction} (a).
The formal definition as an ILP problem is given by \ref{def:maxis}.
\begin{definition}[MaxIS] \label{def:maxis}
    Let \( G=(V, E), V \neq \emptyset \) be an undirected graph, then a maximum independent set problem for the graph \( G \) is the following:
    \begin{align*}
         & \max \sum\limits_{v \in V} x_v                               \\
         & x_u + x_v \leq 1               \qquad \forall \{u, v\} \in E \\
         & x_v \in \{0, 1\}
    \end{align*}
\end{definition}

The MaxIS problem is an NP-hard \cite{pemmaraju2003computational} problem since VertexCover can be reduced to an instance of
this problem. But what is even more interesting for us is that this problem looks very similar to the
our projection ILP problem, except the fact, that set of constraint \eqref{eq:num_proj_const} are
omitted and costs are equal to \( 1 \). Moreover, the relation whether two edges of the graph are adjacent
is not transitive as well as the relation of overlapping \ref{def:overlapping}! That's why we will try to reduce
the maximum independent set problem to an instance of the proposed ILP problem.

Idea of the reduction to the MIS (explanation) (without constraint with nproj)

We will consider that target nodes \( t_u, t_v \in T_{V} \) that correspond to graph's nodes \( u, v \in V \) overlap
if and only if there is an edge in the graph \( G \) between these nodes or \( u = v \).
\[
    t_u \cap t_v \neq \emptyset \Leftrightarrow
    \Biggl[
    \begin{array}{l}
        \{ u, v \} \in E \\
        u = v
    \end{array}
\]
Such a definition of overlapping allow us to match all properties, i.e. reflexivity, symmetricity and non transitivity,
of overlapping relation on word ranges in the initial definition \ref{def:overlapping}. The fact that we consider
overlapping nodes that correspond to the same vertex simply means that it is impossible to
add the same vertex in the independent set to time which is a property of the any set.

\begin{equation} \label{eq:reduction_without_nproj}
    \begin{aligned}
         & \max \sum\limits_{v \in V} x_v                          \\
         & x_u + x_v \leq 1               & \forall \{u, v\} \in E \\
         & x_v \in \{0, 1\}
    \end{aligned}
\end{equation}

\begin{table}[h]
    \begin{subtable}[t]{0.5\linewidth}
        \centering
        \renewcommand{\arraystretch}{1.5}
        \begin{tabular}{|c|c|c|}
            \hline
            \(<\)                                 & \( \leq \), \( = \)               & \(>\), \( \geq \) \\
            \hline
            \( \ceil{\frac{|V|}{n_{proj} - 1}} \) & \( \ceil{\frac{|V|}{n_{proj}}} \) & 1                 \\
            \hline
        \end{tabular}
        \caption{For source nodes}
        \label{tab:src_dummy_nodes_num}
    \end{subtable}
    \begin{subtable}[t]{0.5\linewidth}
        \centering
        \renewcommand{\arraystretch}{1.5}
        \begin{tabular}{|c|c|c|c|}
            \hline
            \(<\), \(\leq\) & \( = \)                                    & \(>\)          & \( \geq \)         \\
            \hline
            \( 0 \)         & \( \ceil{\frac{|V|}{n_{proj}}} n_{proj} \) & \( n_{proj} \) & \( n_{proj} + 1 \) \\
            \hline
        \end{tabular}
        \caption{For target nodes}
        \label{tab:tgt_dummy_nodes_num}
    \end{subtable}
    \caption{Number of dummy nodes for every type of constraints \eqref{eq:num_proj_const}}
    \label{tab:dummy_nodes_num}
\end{table}

What is optimal solution and quotient set of solutions
\[
    T^*_{x} = \{ t \in T_V | \exists s \in S, x_{s, t} = 1 \}
\]

Let's \( x_1, x_2 \) be a two feasible solutions of problem \eqref{eq:reduction_without_nproj}, then we will say that
they are in relation~\( \sim \) if their corresponding solutions of the MaxIS problem are equal:
\[
    x_1 \sim x_2 \Leftrightarrow T^{*}_{x_1} = T^{*}_{x_2}
\]
Since this relation defined by an equality of sets it is reflexive, symmetric and transitive and therefore
equivalence relation.

Then quotient set \( \quot{X}{\sim} \) will consist of all equivalence classes of feasible solutions
that differ only in matching of dummy nodes. Let's notice that the objective function value is equal for all
elements of any equivalence class.
\begin{lemma}
    Cost inside equiv. class are constant
\end{lemma}
\begin{proof}
    Trust me!
\end{proof}

Thus we will call optimal an equivalence class with the highest objective value.

\begin{lemma}
    Let \( x \) be a feasible solution of the problem \eqref{eq:reduction_without_nproj}.
    Then a solution \( x^* \) such that \( t \in T \)
\end{lemma}
\begin{proof}
    Trust me!
\end{proof}

\begin{corollary}
    It is possible to bound sum over all candidates for one source entity
\end{corollary}

\begin{theorem}[]
    The proposed ILP problem \eqref{eq:objective}--\eqref{eq:binary_prog} is NP-hard
\end{theorem}
\begin{proof}
    Trust me!
\end{proof}


\subsection{Approaches to compute the solution of the problem}
greedy algorithm
\begin{algorithm}
    \caption{Greedy algorithm for the proposed ILP problem} \label{alg:ilp_greedy}
\end{algorithm}