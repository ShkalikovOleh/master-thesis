\chapter{Methodology}
\label{sec:methodology}
The overall idea of formalizing the projection step of the XLNER pipeline is
to formulate it as a matching process between source entities and ranges
of words in the target sentence, referred to as target candidates. This process is
illustrated in Figure \ref{fig:cand_matching}. Whereas a predetermined set of source entities
for projection is provided; a subset of continuous word ranges from the original
sentence in the target language can be selected and considered as potential
projections for source entities. The likelihood scores that each source entity
should be projected onto each candidate will be computed. The goal is to identify a
combination of source entities and target candidates that maximizes the overall
sum of the selected tuples. However, natural constraints must also be taken into account.
For instance, when two source entities project onto overlapping candidates, it becomes
ambiguous to which source entity the overlapping words pertain; therefore, projections
onto overlapping candidates should be prohibited. Furthermore, it is logical to
restrict the number of candidates to which each source entity can be projected,
as a one-to-one correspondence between source entities and entities in the
target sentence is typically anticipated. Based on these fundamental properties,
an integer linear optimization problem can be established for further analysis.

\begin{figure*}[ht]
  \centering
  \begin{tikzpicture}[node distance=-0.1,
      every node/.style={text centered,
        text height=2ex,
        text depth=.25ex,
      },
      loc/.style={fill=orange!30, rounded rectangle, label={[anchor=center,font=\tiny\bfseries\sffamily]above:#1-LOC}},
      per/.style={fill=green!30, rounded rectangle, label={[anchor=center,font=\tiny\bfseries\sffamily]above:#1-PER}},
    cand/.style={fill=blue!30, rounded rectangle},]

    \node[per={B}, rounded rectangle east arc=none](George_src){George};
    \node[per={I}, rounded rectangle west arc=none, right=of George_src](Washington_src){Washington};
    \node[right=of Washington_src](is_src){is};
    \node[right=of is_src](the_src){the};
    \node[right=of the_src](first_src){first};
    \node[right=of first_src](president_src){president};
    \node[right=of president_src](of_src){of};
    \node[right=of of_src](the_src){the};
    \node[loc={B}, rounded rectangle east arc=none, right=of the_src](United_src){United};
    \node[loc={I}, rounded rectangle west arc=none, right=of United_src](States_src){States};

    \node[cand, rounded rectangle east arc=none, below=of George_src, yshift=-1.5cm](George_tgt){George};
    \node[cand, rounded rectangle west arc=none, right=of George_tgt](Washington_tgt){Washington};
    \node[right=of Washington_tgt](ist_tgt){ist};
    \node[right=of ist_tgt](der_tgt){der};
    \node[right=of der_tgt](erste_tgt){erste};
    \node[right=of erste_tgt](president_tgt){Präsident};
    \node[right=of president_tgt](der_tgt){der};
    \node[cand, rounded rectangle east arc=none, right=of der_tgt](Vereinigten_tgt){Vereinigten};
    \node[cand, rounded rectangle west arc=none, right=of Vereinigten_tgt](Staaten_tgt){Staaten};

    \node[text=gray, font=\scriptsize, above=of first_src, yshift=0.2cm, xshift=0.2cm](source){Source labeled sentence};
    \node[text=gray, font=\scriptsize, below=of source, yshift=-3cm]{Original sentence with extracted candidates};

    \draw[->] (George_src.south east) -- node[left]{\(c_{11}\)} (George_tgt.north east);
    \draw[->] (United_src.south east) -- node[right]{\(c_{22}\)} (Vereinigten_tgt.north east);
    \draw[->] (George_src.south east) -- node[above left, yshift=0.1cm, xshift=-0.2cm]{\(c_{12}\)} (Vereinigten_tgt.north east);
    \draw[->] (United_src.south east) -- node[below left]{\(c_{21}\)} (George_tgt.north east);
  \end{tikzpicture}
  \caption{Illustration of the proposed idea of matching source entities and candidates in the target sentence}
  \label{fig:cand_matching}
\end{figure*}

\section{Formulation of the ILP problem}
First and foremost, a definition of overlapping candidate should be established.

\begin{definition}[Relation of overlapping] \label{def:overlapping}
  Target candidates \( \tgt{p_1} = (i_{\tgt{p_1}}, j_{\tgt{p_1}}) \) and \linebreak
  \( \tgt{p_2} = (i_{\tgt{p_2}}, j_{\tgt{p_2}}) \in T \) are considered to overlap
  if and only if \( ( i_{\tgt{p_1}} \leq j_{\tgt{p_2}} ) \land ( i_{\tgt{p_2}} \leq j_{\tgt{p_1}} ) \),
  which indicates that the sets of word indices are not disjoint. Overlapping candidates
  will be denoted using the following notation: \( \tgt{p_1} \cap \tgt{p_2} \neq \emptyset \).
\end{definition}

Having defined all necessary objects, the ILP problem for the projection step of the XLNER pipeline that
aligns with our requirements can now be formulated.
Let \( S \) be a set of source entities, \( T \) -- set of target candidates and \( \cap \subset T^2 \) is
a relation of overlapping, then the projection ILP problem is the following:
\begin{align}
  \label{eq:objective}
  & \max\limits_x \sum\limits_{(\src{p}, \tgt{p}) \in S \times T} c_{\src{p}, \tgt{p}} x_{\src{p}, \tgt{p}}                                             \\
  & \text{subject to} \notag                                                                                                                            \\
  \label{eq:num_proj_const}
  & \sum\limits_{\tgt{p} \in T} x_{\src{p}, \tgt{p}} \lessgtr n_{proj}                                      & \forall \src{p} \in S                     \\
  \label{eq:non_overlap_const}
  & x_{\src{p_1}, \tgt{p_1}} + x_{\src{p_2}, \tgt{p_2}} \leq 1
  & \forall (\src{p_1}, \src{p_2}, \tgt{p_1}, \tgt{p_2}) \in \hat{\Pi}(S, T)                                                                            \\
  \label{eq:binary_prog}
  & x_{\src{p}, \tgt{p}} \in \{ 0, 1 \}                                                                     & \forall (\src{p}, \tgt{p}) \in S \times T
\end{align}
where \( \hat{\Pi}(S, T) \) represents a set of combinations of
source entities and target candidates that cannot be projected onto
together due to overlapping. This set is defined as follows:
\begin{equation}  \label{eq:overlapping_set}
  \begin{aligned}
    \hat{\Pi}(S, T) = \Big\{ (\src{p_1}, \src{p_2}, \tgt{p_1}, \tgt{p_2}) \Big| \src{p_1}, \src{p_2} \in S, \tgt{p_1}, \tgt{p_2} \in T, \quad \tgt{p_1} \cap \tgt{p_2} \neq \emptyset, \\
    (\src{p_1} \neq \src{p_2}) \lor (\tgt{p_1} \neq \tgt{p_2}) \Big\}
  \end{aligned}
\end{equation}

Here, each variable \( x_{\src{p}, \tgt{p}} \) indicates whether
the source entity \( \src{p} \in S \) is projected onto the target
candidates \( \tgt{p} \in T \). The set of constraints \eqref{eq:non_overlap_const}
ensures that it is impossible for one or more source entities to be
projected onto overlapping target candidates. Additionally, another
set of constraints \eqref{eq:num_proj_const} limits the number of
projections for each source entity. It should be noted that the
inequality sign is not fixed, as various forms may be relevant: "less,"
"less than or equal," and "equal" can be used to ensure that a source entity
is not projected onto as many available candidates as possible solely to increase
the objective value. Conversely, "greater" or "greater than or equal" can be
employed to enforce that a source entity will be projected at least a specified
number of times, e.g., once, ensuring that the solver does not simply ignore it.
It is evident that \( n_{proj} = 0 \) represents a corner case that typically
lacks meaning, except in the case of a "greater" or "greater or equal" inequality, which effectively
eliminates any limits and is equivalent to having no constraints of this type.

However, the number of constraints \eqref{eq:non_overlap_const} scales as
\( \Theta(n^3 m^2) \) where \( m \) is the the number of source entities and
\( n \) is the length of the target sentence, assuming that all possible continuous subranges are utilized as candidates.
This considerable growth in constraints can complicate the problem-solving process.
Consequently, it is essential to explore methods to reduce the number of such
constraints. To begin, let us examine the overlapping relation that forms the
basis of this issue.
\begin{lemma} \label{lemma:not_transitive}
  The relation of overlapping is not transitive.
\end{lemma}
\begin{proof}
  This will be demonstrated by providing a counterexample.
  Assume the following target candidates:
  \[
    a = (2, 4) \quad b = (1, 2) \quad c = (4, 5)
  \]
  By the definition of the overlapping relation \( a \cap b \neq \emptyset \) and
  \( a \cap c \neq \emptyset \) but \( b \) and \( c \) are not overlapping.
\end{proof}

This fact leads to the conclusion that it is impossible to partition all target
candidates into groups of mutually overlapping candidates and therefore select at most
one candidate from each group to match with a source entity.
% \begin{corollary}
%   It is impossible to partition set of target candidates such that every partition consists of
%   candidates that are pairwise overlapping with each other and there are no overlapping candidates
%   that are in different partitions.
% \end{corollary}
% \begin{proof}
%   By Lemma \ref{lemma:not_transitive} the overlapping relation is not transitive and therefore
%   is not an equivalence relation that implies we can not construct a quotient set.
% \end{proof}

Nevertheless, it is possible to reduce the number of constraints
\eqref{eq:non_overlap_const}. The underlying idea for this reduction is that
if it is impossible to project one or any two source entities onto overlapping
candidates, then it is also impossible to project any number of source entities
onto these candidates. Consequently, it suffices to sum the constraints across
all source entities. This approach is feasible because all variables
\( x_{\src{p}, \tgt{p}} \) are binary, and therefore, they cannot be
negative.
\begin{theorem}
  The set of constraints \eqref{eq:non_overlap_const} is satisfied if and only if
  the following sets of reduced constraints are satisfied:
  \begin{align*}
    & \sum\limits_{\src{p} \in S} (x_{\src{p}, \tgt{p_1}} + x_{\src{p}, \tgt{p_2}}) \leq 1                                & \forall (\tgt{p_1}, \tgt{p_2}) \in \Pi(T) \\
    & \sum\limits_{\src{p} \in S} x_{\src{p}, \tgt{p}} \leq 1
    & \forall \tgt{p} \in T \Big| \nexists \tgt{p_2} \in T: \tgt{p} \neq \tgt{p_2}, \tgt{p} \cap \tgt{p_2} \neq \emptyset                                             \\
  \end{align*}
  where
  \begin{equation*}
    \Pi(T) = \left\{ (\tgt{p_1}, \tgt{p_2}) \Big| \tgt{p_1}, \tgt{p_2} \in T,
      \quad \tgt{p_1} \cap \tgt{p_2} \neq \emptyset,
    \tgt{p_1} \neq \tgt{p_2} \right\}
  \end{equation*}
\end{theorem}
\begin{proof}
  \textit{Necessity:} Assume that constraints \eqref{eq:non_overlap_const} are satisfied, but
  the proposed constraints are not. Then there are two options:
  \[
    \exists (\tgt{p_1}, \tgt{p_2}) \in \Pi(T) \Bigg| \quad
    \sum\limits_{\src{p} \in S} (x_{\src{p}, \tgt{p_1}} + x_{\src{p}, \tgt{p_2}}) > 1
  \]
  or
  \[
    \exists \tgt{p} \in T \Big|
    \nexists \tgt{p_2} \in T: \tgt{p} \neq \tgt{p_2}, \tgt{p} \cap \tgt{p_2} \neq \emptyset
    \Bigg| \quad \sum\limits_{\src{p} \in S} x_{\src{p}, \tgt{p}} > 1
  \]

  Consider the first option:
  \begin{align*}
    & \sum\limits_{\src{p} \in S} (x_{\src{p}, \tgt{p_1}} + x_{\src{p}, \tgt{p_2}}) > 1
    \stackrel{\eqref{eq:binary_prog}}{\implies}                                                                     \\
    & \exists \src{p_1}, \src{p_2} \in S \Big| x_{\src{p_1}, \tgt{p_1}} = 1, x_{\src{p_2}, \tgt{p_2}} = 1 \implies \\
    & x_{\src{p_1}, \tgt{p_1}} + x_{\src{p_2}, \tgt{p_2}} = 2 > 1
  \end{align*}
  This situation contradicts our initial assumption that the constraints \eqref{eq:non_overlap_const} are satisfied.

  The same result we can obtain for the second case:
  \begin{align*}
    & \sum\limits_{\src{p} \in S} x_{\src{p}, \tgt{p}} > 1
    \stackrel{\eqref{eq:binary_prog}}{\implies}                                                                                           \\
    & \exists \src{p_1}, \src{p_2} \in S, \src{p_1} \neq \src{p_2} \Big| x_{\src{p_1}, \tgt{p}} = 1, x_{\src{p_2}, \tgt{p}} = 1 \implies \\
    & x_{\src{p_1}, \tgt{p}} + x_{\src{p_2}, \tgt{p}} = 2 > 1
  \end{align*}

  Thus, the necessity has been established.

  \textit{Sufficiency:}
  Suppose that the following constraints are satisfied:
  \[
    \sum\limits_{\src{p} \in S} (x_{\src{p}, \tgt{p_1}} + x_{\src{p}, \tgt{p_2}}) \leq 1
    \qquad \forall (\tgt{p_1}, \tgt{p_2}) \in \Pi(T)
  \]
  Therefore, the sum can take on values of \( 0 \) or \( 1 \).
  If the sum is zero, then no entities are projected onto any of the overlapping
  candidates, and consequently, the constraints \eqref{eq:non_overlap_const} are
  not violated. Now, let us consider the case when the sum is equal to \( 1 \).
  \begin{equation*} \label{eq:derivation_of_const_reduction}
    \begin{aligned}
      & \sum\limits_{\src{p} \in S} (x_{\src{p}, \tgt{p_1}} + x_{\src{p}, \tgt{p_2}}) = 1
      \stackrel{\eqref{eq:binary_prog}}{\implies}                                                       \\
      & \exists! \src{p} \in S, \exists! \tgt{p} \in \{ \tgt{p_1}, \tgt{p_2}\}
      \Bigg| x_{\src{p}, \tgt{p}} = 1 \implies                                                          \\
      & \forall \src{p_1}, \src{p_2} \in S, x_{\src{p_1}, \tgt{p_1}} + x_{\src{p_2}, \tgt{p_2}} \leq 1
    \end{aligned}
  \end{equation*}
  Therefore we proved that
  \[
    x_{\src{p_1}, \tgt{p_1}} + x_{\src{p_2}, \tgt{p_2}} \leq 1 \qquad
    \forall (\src{p_1}, \src{p_2}, \tgt{p_1}, \tgt{p_2}) \in \hat{\Pi'}(S, T)
  \]
  where
  \begin{align*}
    \hat{\Pi'}(S, T) = \Big\{ (\src{p_1}, \src{p_2}, \tgt{p_1}, \tgt{p_2}) \Big| \src{p_1}, \src{p_2} \in S, \tgt{p_1}, \tgt{p_2} \in T, \quad \tgt{p_1} \cap \tgt{p_2} \neq \emptyset, \\
    \tgt{p_1} \neq \tgt{p_2} \Big\}
  \end{align*}

  But it is not the same as set \eqref{eq:overlapping_set}:
  \[
    \hat{\Pi}(S, T) \setminus \hat{\Pi'}(S, T) = \Big\{ (\src{p_1}, \src{p_2}, \tgt{p_1}, \tgt{p_2}) \Big| \src{p_1}, \src{p_2} \in S, \tgt{p_1}, \tgt{p_2} \in T, \tgt{p_1} = \tgt{p_2},
    \src{p_1} \neq \src{p_2} \Big\}
  \]
  So we need to check whether constraints \eqref{eq:non_overlap_const} are satisfied on this set difference.

  First of all, let's notice that
  \begin{multline*}
    \forall (\tgt{p_1}, \tgt{p_2}) \in \Pi(T)
    \qquad
    \sum\limits_{\src{p} \in S} (x_{\src{p}, \tgt{p_1}} + x_{\src{p}, \tgt{p_2}})
    =
    \sum\limits_{\src{p} \in S} x_{\src{p}, \tgt{p_1}} +
    \sum\limits_{\src{p} \in S} x_{\src{p}, \tgt{p_2}} \leq 1
    \implies                        \\
    \forall \tgt{p} \in T \Bigg| \exists \tgt{p_2} \in T: \tgt{p} \neq \tgt{p_2}, \tgt{p} \cap \tgt{p_2} \neq \emptyset
    \quad
    \sum\limits_{\src{p} \in S} x_{\src{p}, \tgt{p}} \leq 1
  \end{multline*}
  Therefore there exists at most one source entity \( \src{p} \in S \) that are projected
  to the target candidate \( \tgt{p} \).
  And using exactly the same derivation as above we get:
  \begin{multline*}
    \forall \tgt{p} \in T \Bigg| \exists \tgt{p_2} \in T: \tgt{p} \neq \tgt{p_2}, \tgt{p} \cap \tgt{p_2} \neq \emptyset
    \quad
    \sum\limits_{\src{p} \in S} x_{\src{p}, \tgt{p}} \leq 1
    \stackrel{\eqref{eq:binary_prog}}{\implies}                                                                                                                                                       \\
    \begin{aligned}
      & \sum\limits_{\src{p} \in S} x_{\src{p}, \tgt{p}} = 0
      \stackrel{\eqref{eq:binary_prog}}{\implies}
      \forall \src{p} \in S, x_{\src{p}, \tgt{p}} = 0         \\
      \text{or}                                               \\
      & \sum\limits_{\src{p} \in S} x_{\src{p}, \tgt{p}} = 1
      \stackrel{\eqref{eq:binary_prog}}{\implies}
      \exists! \src{p} \in S, x_{\src{p}, \tgt{p}} = 1        \\
    \end{aligned} \implies \\
    x_{\src{p_1}, \tgt{p}} + x_{\src{p_2}, \tgt{p}} \leq 1 \qquad
    \begin{aligned}
      & \forall \src{p_1}, \src{p_1} \in S, \src{p_1} \neq \src{p_2}                                                        \\
      & \forall \tgt{p} \in T \Bigg| \exists \tgt{p_2} \in T: \tgt{p} \neq \tgt{p_2}, \tgt{p} \cap \tgt{p_2} \neq \emptyset
    \end{aligned}
  \end{multline*}
  So, only the case, when entity has no overlapped entity that are
  not equal to it, remains. But this case is fully covered by the second part of the proposed reduced constraints:
  \[
    \sum\limits_{\src{p} \in S} x_{\src{p}, \tgt{p}} \leq 1 \qquad
    \forall \tgt{p} \in T \Big| \nexists \tgt{p_2} \in T: \tgt{p} \neq \tgt{p_2}, \tgt{p} \cap \tgt{p_2} \neq \emptyset
  \]
  And by considering that this sum can be either \( 0 \) or \( 1 \) and making
  exactly the same derivation we conclude that it implies that
  constraints \eqref{eq:non_overlap_const} are satisfied.
\end{proof}

The theorem allows to reduce number of constraints, ensuring that they scale scale as a constant
with respect to the number of source entities. Consequently, we arrive at the following final
formulation of the ILP problem for the projection step of the XLNER pipeline.
\begin{equation} \label{eq:ilp}
  \begin{aligned}
    & \max\limits_x \sum\limits_{(\src{p}, \tgt{p}) \in S \times T} c_{\src{p}, \tgt{p}} x_{\src{p}, \tgt{p}}                                                                                                                       \\
    & \text{subject to}                                                                                                                                                                                                             \\
    & \sum\limits_{\tgt{p} \in T} x_{\src{p}, \tgt{p}} \lessgtr n_{proj}                                      & \forall \src{p} \in S                                                                                               \\
    & \sum\limits_{\src{p} \in S} (x_{\src{p}, \tgt{p_1}} + x_{\src{p}, \tgt{p_2}}) \leq 1                    & \forall (\tgt{p_1}, \tgt{p_2}) \in \Pi(T)                                                                           \\
    & \sum\limits_{\src{p} \in S} x_{\src{p}, \tgt{p}} \leq 1                                                 & \forall \tgt{p} \in T \Big| \nexists \tgt{p_2} \in T: \tgt{p} \neq \tgt{p_2}, \tgt{p} \cap \tgt{p_2} \neq \emptyset \\
    & x_{\src{p}, \tgt{p}} \in \{ 0, 1 \}                                                                     & \forall (\src{p}, \tgt{p}) \in S \times T
  \end{aligned}
\end{equation}

\section{Candidates extraction}
Whereas the set \( S \) of source entities is given, as all source entities have been
extracted in the previous step of the XLNER pipeline, the construction of the set
\( T \) of target candidates remains an open question.

The simplest method for candidate extraction is to consider all possible n-grams
(continuous ranges of words) from the target sentence as candidates. This approach
guarantees that no actual target entity will be excluded from the set of target
candidates. However, from a computational perspective, this can pose a challenge,
as the number of all n-grams scales quadratically. Furthermore, from an application
standpoint, the majority of candidates generated in this manner are unlikely to
represent valid target entities.

One of the simplest strategies to address this problem is to limit the maximum
possible length of the candidates. From the perspective of Named Entity Recognition,
it is reasonable to expect that actual target entities typically do not exceed a
certain predefined length. For instance, it is unlikely for a person's name to consist
of more than 10 words, and if there are three source entities, it is unlikely
that there exists a single target entity that consists of all words of the target
sentence.

Thus, the algorithm for candidates extraction wilt a bounded maximum length of n-grams
is the following.
\begin{algorithm}
  \KwData{\( n \in \mathbb{N} \) -- number of words in the target sentence,
  \(M \leq n \in \mathbb{N} \) -- maximum length of a target candidate}
  \KwResult{\( T \) -- set of target candidates}

  \( T \gets \emptyset \) \;
  \For{\(i \gets 0 \) \KwTo \( n \)}{
    \( m \gets \min(s + M, n) \) \;
    \For{\(j \gets i \) \KwTo \( m \)}{
      \( T \gets T \cup \{ (i, j) \} \) \;
    }
  }
  \caption{Bounded length n-gram candidates extraction}
  \label{alg:ngram_extraction}
\end{algorithm}

An alternative approach for candidate extraction is to employ a model to generate
candidates. In this regard, TProjection \cite{garcia-ferrero-etal-2023-projection} utilizes a fine-tuned T5 model with beam
search for candidate generation. In theory, any large language model can be
fine-tuned for this purpose. However, the primary drawback of this approach is
the necessity to fine-tune these models, which requires a labeled dataset in
the target language or relies on cross-lingual model transfer. In this context,
it may be more feasible and effective to train a model specifically to label target
entities rather than to generate candidates for the projection step of the
XLNER pipeline. Moreover, autoregressive models tend to be resource-intensive,
making it inefficient to utilize them solely for candidate extraction.

It also makes sense to consider a multilingual encoder-only Transformer \cite{vaswani2017attention}
model with high recall. However, these models still require training to extract candidates
effectively. In the following chapters, these alternatives will not be discussed
and will remain as topics for future work.

\section{Matching scores}
Thus far, the ILP problem \eqref{eq:ilp} has been formulated and discussed,
which involves projecting source entities onto target candidates using a matching
score \( c_{\src{p}, \tgt{p}} \). This score represents the likelihood that a given
source entity \( \src{p} \in S \) should be projected onto the corresponding target
candidate \( \tgt{p} \in T \). However, the question of how to compute all these
scores remains unresolved. In this section, various options for evaluating these
scores will be proposed.

\subsection{Alignment-based score}
In Chapter \ref{sec:background}, it was demonstrated that word-to-word alignments
can be utilized for the projection step of the XLNER pipeline. However this was
achieved through a heuristic algorithm. Nonetheless, an attempt can be made to
incorporate these alignments into the proposed ILP problem by calculating matching
scores using word-to-word alignments.

The matching score that inspired from this idea is the following:
\begin{equation} \label{eq:align_cost}
  c_{\src{p}, \tgt{p}}^{align} =
  \frac{\sum\limits_{k=i_{\src{p}}}^{j_{\src{p}}} \sum\limits_{l=i_{\tgt{p}}}^{j_{\tgt{p}}} a_{kl}}
  {(j_{\src{p}} - i_{\src{p}}) + (j_{\tgt{p}} - i_{\tgt{p}})}
\end{equation}

The form of the score is grounded in the natural properties expected from a score
based on word-to-word alignments. First, the greater the number of aligned words
between the source entity and the target candidates, the higher the score should be.
Second, it is insufficient to simply divide the number of aligned words by the length
of the source entity, as this would lead to a higher score for longer target candidates.
Lastly, the third property is particularly beneficial: when considering two candidates,
where one is a substring of the other, if the total number of aligned words between the
source entity and these candidates is the same, preference should be given to the
smaller candidate. The score defined in equation \eqref{eq:align_cost} fulfills this
requirement.
\begin{lemma} \label{lemma:align_cost_decrease}
  Suppose that for a specific pair of source entity \( \src{p} \in S \) and
  target candidate \( \tgt{p} \in T \), the score defined in equation
  \eqref{eq:align_cost} is equal to \( c \). If there exists an extended
  candidate \( \tgt{\hat{p}} \in T \) that is increased by one word to the
  left or right, such that this additional word is not aligned with any word
  of the source entity, then the score \eqref{eq:align_cost} between the source
  entity and this target candidate will be lower than \( c \).
\end{lemma}
\begin{proof}
  Considering the extension to the right, let \( \tgt{\hat{p}} = (i_{\tgt{p}}, j_{\tgt{p}} + 1) \).
  Given that this additional word is not aligned with any word of the source entity,
  it implies that:
  \[
    \sum\limits_{k=i_{\src{p}}}^{j_{\src{p}}} a_{k,j_{\tgt{p}} + 1} = 0
  \]
  Then:
  \begin{multline*}
    c_{\src{p}, \tgt{\hat{p}}}^{align} =
    \frac{\sum\limits_{k=i_{\src{p}}}^{j_{\src{p}}} \sum\limits_{l=i_{\tgt{p}}}^{j_{\tgt{p}} + 1} a_{kl}}
    {(j_{\src{p}} - i_{\src{p}}) + (j_{\tgt{\hat{p}}} - i_{\tgt{\hat{p}}})} =                                \\
    \frac{\sum\limits_{k=i_{\src{p}}}^{j_{\src{p}}} \sum\limits_{l=i_{\tgt{p}}}^{j_{\tgt{p}}} a_{kl}}
    {(j_{\src{p}} - i_{\src{p}}) + (j_{\tgt{p}} + 1 - i_{\tgt{p}})} +
    \frac{\sum\limits_{k=i_{\src{p}}}^{j_{\src{p}}} a_{k,j_{\tgt{p}} + 1}}
    {(j_{\src{p}} - i_{\src{p}}) + (j_{\tgt{p}} + 1 - i_{\tgt{p}})} =                                        \\
    \frac{\sum\limits_{k=i_{\src{p}}}^{j_{\src{p}}} \sum\limits_{l=i_{\tgt{p}}}^{j_{\tgt{p}}} a_{kl}}
    {(j_{\src{p}} - i_{\src{p}}) + (j_{\tgt{p}} - i_{\tgt{p}}) + 1} <
    \frac{\sum\limits_{k=i_{\src{p}}}^{j_{\src{p}}} \sum\limits_{l=i_{\tgt{p}}}^{j_{\tgt{p}}} a_{kl}}
    {(j_{\src{p}} - i_{\src{p}}) + (j_{\tgt{p}} - i_{\tgt{p}})} = c
  \end{multline*}
  The derivation for the extension to the left, where
  \( \tgt{\hat{p}} = (i_{\tgt{p}} - 1, j_{\tgt{p}}) \), is structurally analogous.
  Thus, the lemma has been proven.
\end{proof}

\begin{corollary} \label{col:shrink_cand}
  The cost of the extended candidate \( \tgt{p_{+n}} \) derived from the target
  candidate \( \tgt{p} \), when extended to the left or right by
  \( n \in \mathbb{N} \) non-aligned words that do not correspond to any word
  of the source entity \( \src{p} \in S \), is lower than the cost of the original
  candidate.
\end{corollary}
\begin{proof}
  Let us prove this statement using mathematical induction for the case of extension to the right (the proof for the left extension will follow a similar structure).

  \textit{Base:} By the lemma \ref{lemma:align_cost_decrease} \( c_{\src{p}, \tgt{p}}^{align} > c_{\src{p}, \tgt{p_{+1}}}^{align} \).

  \textit{Induction step:} Assume that for some \( k \in \mathbb{N} \) holds \( c_{\src{p}, \tgt{p}}^{align} > c_{\src{p}, \tgt{p_{+k}}}^{align} \).
  Then by the lemma~\ref{lemma:align_cost_decrease} \( c_{\src{p}, \tgt{p_{+k}}}^{align} > c_{\src{p}, \tgt{p_{+(k+1)}}}^{align} \) and therefore
  \( c_{\src{p}, \tgt{p}}^{align} > c_{\src{p}, \tgt{p_{+(k+1)}}}^{align} \).

  Thus, by the mathematical induction, \( c_{\src{p}, \tgt{p}}^{align} > c_{\src{p}, \tgt{p_{+n}}}^{align}   \)
\end{proof}

The main advantage of this property is that it allows for the reduction of the set of
target candidates without compromising optimal solutions. Specifically, it permits
the consideration of candidates only from the subrange between the leftmost and
rightmost words aligned with any word of any source entity.
\begin{theorem}
  Let \( m, M \in \mathbb{N} \) denote the indices of the leftmost and rightmost
  aligned words corresponding to any source entity. Then there exists
  an optimal solution for the problem \eqref{eq:ilp} where the constraints
  \eqref{eq:num_proj_const} take the form of \( < \) or \( \leq \), such that
  all source entity are projected onto a target candidates from the set \( T \)
  generated by the algorithm \ref{alg:ngram_extraction}, where all word indices
  are contained within the range \( [m, M] \).
\end{theorem}
\begin{proof}
  Suppose that there is no optimal solution that satisfies this requirement.
  This implies the existence of a source entity \( \src{p} \in S \) and a target
  candidate \( \tgt{\hat{p}} \in T \) such that \( (i_{\tgt{\hat{p}}} < m) \lor (M < j_{\tgt{\hat{p}}}) \)
  and \( x_{\src{p}, \tgt{\hat{p}}} = 1 \) for an optimal solution \( x \).

  This case can be divided into two parts: one where \( \tgt{\hat{p}} \) contains
  no words aligned with any words of the source entities, and the other where it
  does have such alignments.

  Considering the first scenario, if \( \tgt{\hat{p}} \) consists entirely of
  words that are not aligned with any word from the source entities, then the
  alignment-based matching cost, according to the definition in \eqref{eq:align_cost},
  equals \( 0 \).

  In this situation, we can take a solution \( x^* \) that is identical to the
  optimal solution \( x \), except for the target candidate \( \tgt{\hat{p}} \);
  that is, we set \( x_{\src{p}, \tgt{\hat{p}}} = 0 \). Consequently, the objective
  function remains unchanged because the cost of matching is zero. Additionally,
  we do not violate the non-overlapping constraints \eqref{eq:non_overlap_const}:
  \begin{align*}
    & \forall \src{p_1} \in S, \tgt{p_1} \in T \Big| (\src{p_1}, \src{p}, \tgt{p_1}, \tgt{\hat{p}}) \in \hat{\Pi}(S, T) \\
    & x_{\src{p_1}, \tgt{p_1}} + x_{\src{p}, \tgt{\hat{p}}} =
    x_{\src{p_1}, \tgt{p_1}} + 1 \leq 1 \implies                                                                         \\
    & x^*_{\src{p_1}, \tgt{p_1}} + x^*_{\src{p}, \tgt{\hat{p}}} \leq 1 =
    x_{\src{p_1}, \tgt{p_1}} + 0 \leq 1
  \end{align*}
  As well as constraints \eqref{eq:num_proj_const}:
  \begin{multline*}
    \sum\limits_{t \in T} x_{\src{p}, t} \leq n_{proj} \implies                             \\
    \sum\limits_{t \in T} x^*_{\src{p}, t} =
    \sum\limits_{t \in T \setminus \{ \tgt{p} \}} x_{\src{p}, t} + x^*_{\src{p}, \tgt{p}} =
    \sum\limits_{t \in T \setminus \{ \tgt{p} \}} x_{\src{p}, t} + 0 \leq n_{proj}
  \end{multline*}
  Therefore, such a solution \( x^* \) is also optimal, just like the original solution \( x \),
  but with target candidates constrained to have indices within the range \( [m, M] \).

  Considering the second option, where a target candidate \( \tgt{\hat{p}} \) contains words
  that are aligned with words from a source entity \( \src{p} \), i.e.,
  \( \left[ (i_{\tgt{\hat{p}}} < m) \lor (M < j_{\tgt{\hat{p}} < m}) \right] \land
  \left[ (i_{\tgt{\hat{p}}} \leq M) \lor m \leq j_{\tgt{\hat{p}}} \right] \).
  By corollary \ref{col:shrink_cand}, the candidate \( \tgt{p} = ( \max(i_{\tgt{\hat{p}}}, m), \min(M, j_{\tgt{\hat{p}}}) ) \),
  which is a substring of \( \tgt{\hat{p}} \), will have a higher alignment-based
  matching score.

  Let us consider a solution \( x^* \) that is identical to the solution \( x \)
  everywhere except for the target candidates \( \tgt{\hat{p}} \) and \( \tgt{p} \);
  specifically, \( x^*_{\src{p}, \tgt{\hat{p}}} = 0 \) and \( x^*_{\src{p}, \tgt{p}} = 1 \).
  This solution also satisfies all constraints of the ILP problem.

  For the constraints \eqref{eq:num_proj_const}, the proof is as follows:
  \begin{align*}
    & \sum\limits_{t \in T} x_{\src{p}, t} =
    \sum\limits_{t \in T \setminus \{ \tgt{p}, \tgt{\hat{p}} \}} x_{\src{p}, t} + x_{\src{p}, \tgt{p}} + x_{\src{p}, \tgt{\hat{p}}}  =     \\
    & \sum\limits_{t \in T \setminus \{ \tgt{p}, \tgt{\hat{p}} \}} x_{\src{p}, t} + 0 + 1
    \leq n_{proj} \implies                                                                                                                 \\
    & \sum\limits_{t \in T} x^*_{\src{p}, t} =
    \sum\limits_{t \in T \setminus \{ \tgt{p}, \tgt{\hat{p}} \}} x_{\src{p}, t} + x^*_{\src{p}, \tgt{p}} + x^*_{\src{p}, \tgt{\hat{p}}}  = \\
    & \sum\limits_{t \in T \setminus \{ \tgt{p}, \tgt{\hat{p}} \}} x_{\src{p}, t} + 1 + 0 \leq n_{proj}
  \end{align*}
  And for non-overlapping constraints, since \( \tgt{p} \) is a substring of \( \tgt{\hat{p}} \) by a
  construction, we have:
  \[
    \forall t \in T \quad t \cap \tgt{p} \neq \emptyset \implies t \cap \tgt{\hat{p}} \neq \emptyset
  \]
  It gets us that all constraints that should be hold for \( \tgt{p} \) should also be satisfied for \( \tgt{\hat{p}} \):
  \begin{equation} \label{eq:overlap_substring}
    \left\{ (\src{p_1}, \tgt{p_1}) \Big| (\src{p_1}, \src{p}, \tgt{p_1}, \tgt{p}) \in \hat{\Pi}(S, T) \right\} \subset
    \left\{ (\src{p_1}, \tgt{p_1}) \Big| (\src{p_1}, \src{p}, \tgt{p_1}, \tgt{\hat{p}}) \in \hat{\Pi}(S, T) \right\}
  \end{equation}
  This implies that all constraints that must hold for \( \tgt{p} \) should also be satisfied for \( \tgt{\hat{p}} \):
  \begin{align*}
    & \exists \src{p_1} \in S, \tgt{p_1} \in T \Big| (\src{p_1}, \src{p}, \tgt{p_1}, \tgt{p}) \in \hat{\Pi}(S, T) \\
    & x^*_{\src{p_1}, \tgt{p_1}} + x^*_{\src{p}, \tgt{p}} =
    x_{\src{p_1}, \tgt{p_1}} + 1 > 1 \implies                                                                      \\
    & x_{\src{p_1}, \tgt{p_1}} = 1
    \implies                                                                                                       \\
    & x_{\src{p_1}, \tgt{p_1}} + x_{\src{p}, \tgt{\hat{p}}} = 1 + 1 = 2 > 1
  \end{align*}
  However, due to \eqref{eq:overlap_substring}, it follows that
  \( (\src{p_1}, \src{p}, \tgt{p_1}, \tgt{\hat{p}}) \in \hat{\Pi}(S, T) \).
  This contradiction indicates that the solution \( x \) was feasible.
  Consequently, the constraints \eqref{eq:non_overlap_const} are satisfied by the
  solution \( x^* \).

  At the same time, the solution \( x^* \) has a higher objective value, which
  implies that the original solution \( x \) was not optimal. This leads to a
  contradiction with our assumption that there is no optimal solution that satisfies
  the requirements of the theorem.
\end{proof}
It is important to note that the theorem applies only to the constraints
\eqref{eq:num_proj_const} in the forms of \( < \) or \( \leq \). This is due to
the fact that the ILP problem may lose feasibility if projections with zero
scores—those that exist solely to satisfy these constraints and have no practical
application—are removed.

In comparison to the heuristic algorithm that employs word-to-word alignment,
the formulation of the optimization problem with matching scores \eqref{eq:align_cost}
presents a significant advantage: it allows for the evaluation of the confidence
associated with each specific projection, as a score is available for each one.
In contrast, heuristics only provide a labeling of the target sentence. Additionally,
the predictions made by this formulation and the heuristics are not always the same.
For example, the heuristic algorithm merges all continuous ranges of aligned words
that are separated by at most a fixed number of non-aligned words, whereas the
ILP formulation will only do so under certain specific conditions.

It is also worth mentioning that, for the alignment-based score, there is no fixed
upper bound, as the maximum value of the denominator is the product of the lengths of
the source entity and the target candidate. This limitation may be addressed by
replacing one of the summation operations in the denominator of equation
\eqref{eq:align_cost} with a logical OR. Nevertheless, in our experiments, we will
utilize the original form of the alignment-based matching score.

\subsection{NER model-based score}

An other way to evaluate matching scores between source entities and target candidates is to
use a multilingual NER model. The idea is the following: a model for every word of the target candidate
predicts probability distribution over a set of classes that a word has a specific label. In order to compute
a matching score that a source entity with a class \( l \) should be projected onto a target candidate,
we compute an average probability of the class \( l \) over all words of the candidate.

Let \( L \) be a set of classes. Since a NER model predicts output in the IOB format, i.e. the first word of a predicted
entity with class \( l \in L \) has a label \textit{B-l} and other words of this entity have labels \textit{I-l},
the output of a model can be represented as a matrix \( p_{m, o} \) where \( m \in \mathbb{N} \) is an index of a word and
\( o \in \{ 1, 2|L| + 1 \} \) is an index of a label (assume that the label \( O \), i. e. a word doesn't belong to any class,
has an index \( 2|L| + 1 \)).

Then let's introduce mappings \( B[l]: L \rightarrow \{ 1, \dots, 2|L| \} \) and
\( I[l]: L \rightarrow \{ 1, \dots, 2|L| \} \) that returns index of B and I labels respectively
for a class \( l \in L \) in a probability matrix.

So the NER model-based matching score is the following:
\begin{equation} \label{eq:ner_cost}
  c_{\src{p}, \tgt{p}}^{ner} = \alpha^{(j_{\tgt{p}} - i_{\tgt{p}}) - 1}
  \frac{
    p_{i_{\tgt{p}}, B[l_{\src{p}}]} +
    \sum\limits_{k = i_{\tgt{p}} + 1}^{j_{\tgt{p}}} p_{k, I[l_{\src{p}}]}
  }
  {j_\tgt{p} - i_{\tgt{p}}}
\end{equation}
where \( l_{\src{p}} \in L \) is a class of the source entity \( \src{p} \in S \) and
\( \alpha > 0 \in \mathbb{R} \) is a length-scaling constant.

The factor \( \alpha \) is necessary to align matching scores with NER model predictions.
We naturally expect from the matching cost that if a target candidate is a substring of a predicted
by a NER model entity, then its score should be lower that a score of the target candidate that
equal to the predicted entity. For example, let the set of classes contains only class \textit{PER}: \( L = \{ PER \}\),
set of source entities consists of only one entity \( \src{p} \) with a class \textit{PER},
set of target candidates be \( T =  \{ \tgt{p_1} = (1, 1), \tgt{p_2} = (1, 2) \} \) and a probability matrix predicted by the model is:
\[
  \begin{blockarray}{cccc}
    & \text{\textit{B-PER}} & \text{\textit{I-PER}} & \text{\textit{O}} \\
    \begin{block}{c(ccc)}
      1 & 0.9 & 0.1 & 0 \\
      2 & 0.2 & 0.8 & 0 \\
      3 & 0   & 0   & 1 \\
    \end{block}
  \end{blockarray}
\]
By taking maximum over rows it turns out that the entity predicted by a model has a class \textit{PER} and
consists of words with indices 1 and 2. But the NER model-based matching score without scaling factor
\( \alpha \) gives the following results:
\[
  c_{\src{p}, \tgt{p_1}}^{ner} = 0.9 \qquad c_{\src{p}, \tgt{p_2}}^{ner} = 0.85
\]
And in this case the ILP problem will prefer projecting source entity onto a substring of the entity predicted
by the model. Therefore, length-scaling constant \( \alpha \) can not be omitted.

If \( \alpha \) is a crucial part of the score then it make sense to evaluate the range of its values.
For this let's notice the following fact.
\begin{lemma} \label{lemma:maxprob_greater_than_avg}
  Let \( p_{i}, i \in K \) be a probability distribution over a finite set \( K \) with cardinality \( k \).
  Then
  \[
    \max\limits_{i \in K} p_i \geq \frac{1}{k}
  \]
\end{lemma}
\begin{proof}
  Suppose that it is wrong, i.e.
  \[
    \max\limits_{i \in K} p_i < \frac{1}{k}
  \]
  Then let's sum up all probabilities:
  \[
    \sum\limits_{i \in K} p_i \leq \sum\limits_{i \in K} \max\limits_{j \in K} p_j =
    k \cdot \max\limits_{j \in K} p_j < 1
  \]
  But the sum of all probabilities of the distribution should be equal to \( 1 \), therefore our assumption was
  wrong that proofs the lemma.
\end{proof}
Having this useful fact let's finally make some estimation of the constant \( \alpha \).
\begin{theorem}
  Consider a source entity \( \src{p} \in S \) and two target candidates \( \tgt{p_1} = (i, j+1), \tgt{p_2} = (i, j) \).
  Suppose the predictions \( p \) of a NER model satisfy the following constraints:
  \[
    B[l_{\src{p}}] = \argmax\limits_{l \in \{ 1, \dots, 2|L| + 1 \}} p_{i, l} \qquad
    I[l_{\src{p}}] = \argmax\limits_{l \in \{ 1, \dots, 2|L| + 1 \}} p_{k, l} \; \forall k \in \{ i + 1, \dots ,j +1 \}
  \]
  i.e. all these candidates are parts of the predicted by the model entity.
  Let's denote \( M = \frac{1}{2|L| + 1} \).
  Then:
  \[
    \alpha > 1 + \frac{1 - M}{1 + M} \implies c_{\src{p}, \tgt{p_2}}^{ner} > c_{\src{p}, \tgt{p_1}}^{ner}
  \]
\end{theorem}
\begin{proof} Let's write down a desired inequality
  \begin{align*}
    & c_{\src{p}, \tgt{p_2}}^{ner} > c_{\src{p}, \tgt{p_1}}^{ner} \Leftrightarrow \\
    & \alpha^{n}
    \frac{
      p_{i, B[l_{\src{p}}]} +
      \sum\limits_{k = i + 1}^{j + 1} p_{k, I[l_{\src{p}}]}
    }
    {n + 1} >
    \alpha^{n - 1}
    \frac{
      p_{i, B[l_{\src{p}}]} +
      \sum\limits_{k = i + 1}^{j} p_{k, I[l_{\src{p}}]}
    }
    {n}
    \Leftrightarrow                                                                \\
    & \alpha >
    \frac
    {(n+1) \cdot \left( p_{i, B[l_{\src{p}}]} +
    \sum\limits_{k = i + 1}^{j} p_{k, I[l_{\src{p}}]} \right)}
    {
      n \cdot \left( p_{i, B[l_{\src{p}}]} +
    \sum\limits_{k = i + 1}^{j + 1} p_{k, I[l_{\src{p}}]} \right)}
    =
    \frac
    {(n+1) \cdot \left( p_{i, B[l_{\src{p}}]} +
    \sum\limits_{k = i + 1}^{j} p_{k, I[l_{\src{p}}]} \right)}
    {
      n \cdot \left( p_{i, B[l_{\src{p}}]} +
    \sum\limits_{k = i + 1}^{j} p_{k, I[l_{\src{p}}]} + p_{j+1, I[l_{\src{p}}]} \right)}
  \end{align*}
  In order to simplify derivation we will introduce aliases for some quantities:
  \[
    S = p_{i, B[l_{\src{p}}]}  + \sum\limits_{k = i + 1}^{j} p_{k, I[l_{\src{p}}]} \qquad
    s = p_{j+1, I[l_{\src{p}}]}
  \]
  Note, that by the lemma \ref{lemma:maxprob_greater_than_avg} the following
  inequalities holds:
  \begin{equation} \label{eq:ner_sum_ineq}
    M \leq s \leq 1 \qquad nM \leq S \leq n
  \end{equation}
  Then we have:
  \begin{equation*}
    \alpha > \frac{(n + 1) S}{n (S + s)} =
    \frac{nS + S + ns - ns}{nS + ns} =
    1 + \frac{S - ns}{n(S + s)}
  \end{equation*}
  Let's compute derivatives of this expression.
  \begin{equation*}
    \left( 1 + \frac{S - ns}{n(S + s)} \right)_s^{'} =
    \frac{-n^2S - nS}{n^2 (S + s)^2} < 0
    \qquad
    \left( 1 + \frac{S - ns}{n(S + s)} \right)_S^{'}
    = \frac{nS + n^2s}{n^2 (S + s)^2} > 0
  \end{equation*}
  Hence the maximum of the function is located at the point
  \( s = M \) and \( S = n \), then we can take \( \alpha \) such that
  \begin{equation*}
    \alpha > 1 + \frac{1 - M}{1 + M}
    \stackrel{n \geq 1}{\geq}
    1 + \frac{n - nM}{n(n + M)}
    \stackrel{\max}{\geq}
    1 + \frac{S - ns}{n(S + s)}
  \end{equation*}
\end{proof}
As it is shown in the proof it is just an upper estimation of such an alpha for the worst possible case,
in practice there is a sense to try even smaller \( \alpha \), but then it doesn't guarantee that the
desired property are satisfied.

The main difference between a plain model transfer and XLNER pipeline with the ILP projection problem that uses NER model-based
matching cost is that during latter we project labels from source entities. Therefore some, possibly wrong, predictions of the
NER model could be ignored since in the solution of the ILP problem there is no source entity that projects
to this target entity.

One of the problem when it comes to using some model is the fact that models tend to be overconfident in their
predictions, i.e. outputs high probabilities for every, sometimes even wrong, prediction. And since the
proposed score directly uses these probabilities it can affects it. In order to overcome this issue the
calibration of the model should be applied. The simplest approach to do it is to scale output's logits of the
model by some temperature. Nevertheless, in further chapter we will use the NER model without any calibration.

The NER model-based matching score \eqref{eq:ner_cost} also has another property that is worth to mention.
Since the computation of the score involves only the class of the source entity, scores will be equal
for all source entities that have the same class. And from the point of view of the ILP problem
\eqref{eq:ilp} it can lead to a solution where a source entity is projected onto a semantically wrong candidate, but
nevertheless a correct class will be assigned to the target candidate.

\subsection{Translation-based score}

\begin{equation} \label{eq:nmt_cost}
  c_{\src{p}, \tgt{p}}^{nmt}
\end{equation}

\subsection{Fused score}
The performance of alignment, NER and translation models that were used for computation of
scores above varies depending on the language, domain, set of classes, etc.
Every matching score has their advantages and drawbacks, but the natural way to minimize fails is
to fuse scores together. Since all scores are real numbers one of the approaches to do it is to compute a
weighted sum of scores of all types:
\begin{equation} \label{eq:fused_cost}
  c_{\src{p}, \tgt{p}}^{fused} =
  \lambda_{align} c_{\src{p}, \tgt{p}}^{align} +
  \lambda_{ner} c_{\src{p}, \tgt{p}}^{ner} +
  \lambda_{nmt} c_{\src{p}, \tgt{p}}^{nmt}
\end{equation}
where \( \lambda_{align} \geq, \lambda_{ner} \geq 0, \lambda_{nmt} \geq 0 \in R\).

Moreover, the fused score can extend possibilities to handle issues of basic scores.
A NER model that is used in the NER-based cost \eqref{eq:ner_cost} sometimes can properly
predict spans of entities in the target sentence, but confuse classes.
We assume that labels are predicted correctly during computation of the NER model-based cost because
we have no way to correct them. But when the cost consists of different basic score, we can leverage a
NER model-based score only evaluate how likely a given candidate can form a target entity of any class and
determine label by using other types of scores. The modified \eqref{eq:ner_cost} that implements it
is the following:
\begin{equation} \label{eq:ner_cost_wo_classes}
  c_{\src{p}, \tgt{p}}^{ner} = \alpha^{(j_{\tgt{p}} - i_{\tgt{p}}) - 1}
  \max\limits_{l \in L}
  \frac{
    p_{i_{\tgt{p}}, B[l]} +
    \sum\limits_{k = i_{\tgt{p}} + 1}^{j_{\tgt{p}}} p_{k, I[l]}
  }
  {j_\tgt{p} - i_{\tgt{p}}}
\end{equation}

\section{Analysis of the ILP problem}
Despite the fact that the proposed ILP problem is just a particular case
of well-studied the general binary programming problem it doesn't imply that
it inherits all properties of it. For example, maximum bipartite matching problem
can be formulated as an instance of the ILP problem, but still can be solved in a
polynomial time, whereas general case of ILP is NP-hard. That motivates a deeper analysis
of the proposed formulation.

\subsection{Complexity}
Constraints play a crucial role in a complexity of the proposed problem. In the case if all candidates are not overlapping and constraints
\eqref{eq:num_proj_const} has a form of equalities with \( n_{proj} = 1 \) the problem reduces to
an instance of the weighted bipartite matching problem \cite{pemmaraju2003computational}, that is in complexity class \( P \).
But let's study more general case.

For the sake of convenience we will use the first form \eqref{eq:objective}--\eqref{eq:binary_prog}
of the ILP problem in this section, but since it equivalent to the \eqref{eq:ilp} it doesn't affect
the results of the analysis.

The usual way to prove the complexity of some problem is to reduce other problem with known complexity to
an instance of the studied one. Thus let's consider the maximum weigh independent set problem \cite{pemmaraju2003computational}.

\begin{figure}[ht]
  \centering
  \begin{tikzpicture}[every node/.style = {draw, circle}]
    \node[fill=green!10] (1) {1};
    \node[right=of 1] (2) {1};
    \node[below right=of 2, fill=green!10] (3) {1};
    \node[below=of 1, fill=green!10] (4) {3};
    \node[left=of 4] (5) {2};
    \node[below right=of 4] (6) {2};

    \graph{
      (1) -- (2) -- (4) -- (6) -- (5) -- (4),
      (6) -- (2) -- (3)
    };
  \end{tikzpicture}
  \caption{Maximum weigh independent set problem (MaxWIS). Vertices that form an optimal solution are colored in \textbf{\textcolor{green!50}{green}}}
  \label{fig:maxwis}
\end{figure}

The maximum weigh independent set problem (MaxWIS) is a problem of finding a subset of vertices of some undirected
graph with maximum cardinality, such that it doesn't contain any vertices connected by edges of the graph.
An example of the maximum weigh independent set problem is depicted on the figure \ref{fig:maxwis_reduction}.
The formal definition as an ILP problem is given by \ref{def:maxwis}.
\begin{definition}[MaxWIS] \label{def:maxwis}
  Let \( G=(V, E, w), V \neq \emptyset, \; w: V \rightarrow \mathbb{R} \) be an undirected weighted graph, then a maximum weigh
  independent set problem for the graph \( G \) is the following:
  \begin{align*}
    & \max \sum\limits_{v \in V} w_v x_v                               \\
    & x_u + x_v \leq 1               \qquad \forall \{u, v\} \in E \\
    & x_v \in \{0, 1\}
  \end{align*}
\end{definition}

In the case when all weighs are equal to \( 1 \) the problem called a maximum independent set (MaxIS) problem and
the objective is equal to a cardinality of the independent set with maximum number of vertices.

It can be showed, that the generalized form the the projection ILP problem
\eqref{eq:objective}--\eqref{eq:binary_prog} where the relation \( \cap \) is an arbitrary
reflexive, symmetric, non-transitive relation is NP-hard. The reduction of the maximum independent
set problem to the instance of the generalized ILP problem that proved it is given in the
Appendix~\ref{sec:gen_ilp_is_np_hard}. But unfortunately, the overlapping relation \ref{def:overlapping}
of word ranges that is used in the projection ILP problem is a particular case of overlapping relation
for which induced graph of the MaxIS problem is an interval graph, i.e. reduction from the proof
is no longer valid. And it is known that for interval graphs the maximum independent
set problem can be solver in a polynomial time \cite{bhattacharya2014maximum}.

Nevertheless, this results motivates the further analysis. It turns out that the maximum weight
independent set problem on interval graphs can also be solved in a polynomial time \cite{PalB96}.
So, we can prove that the non-overlapping constrains itself don't make the whole projection
ILP problem \eqref{eq:objective}--\eqref{eq:binary_prog} hard.

Consider the projection ILP problem without constraints \eqref{eq:num_proj_const}:
\begin{equation} \label{eq:ilp_without_nproj}
  \begin{aligned}
    & \max\limits_x \sum\limits_{(\src{p}, \tgt{p}) \in S \times T} c_{\src{p}, \tgt{p}} x_{\src{p}, \tgt{p}}                                             \\
    & \text{subject to}                                                                                                                                   \\
    & x_{\src{p_1}, \tgt{p_1}} + x_{\src{p_2}, \tgt{p_2}} \leq 1
    & \forall (\src{p_1}, \src{p_2}, \tgt{p_1}, \tgt{p_2}) \in \hat{\Pi}(S, T)                                                                            \\
    & x_{\src{p}, \tgt{p}} \in \{ 0, 1 \}                                                                     & \forall (\src{p}, \tgt{p}) \in S \times T
  \end{aligned}
\end{equation}
It is possible to reduce this problem to an instance of the MaxWIS problem on an
interval graph.

The idea of the reduction is straightforward. For every target candidate vertex \( t_v \in T \) of the MaxIS problem
we will create one distinct vertex \( v \in V \) that corresponds to this candidate.
The weight of every vertex \( v \) is computed based on matching cost between any source entity and
target candidate that corresponds to this vertex:
\[
  w_v = \max\limits_{s \in S} c_{s, t_v}
\]
And finally let's link solution of the ILP problem \eqref{eq:ilp_without_nproj}
to the solutions of the MaxIS problem:
\begin{equation} \label{eq:link_variables}
  \begin{aligned}
    &s^*_v = \argmax\limits_{s \in S} x_{s, t_v} \\
    &
    \begin{cases}
      x_{s^*, t_v} = 1 \\
      x_{s, t_v} = 0 \quad \forall s \in S \setminus \{ s^* \} \\
    \end{cases}
    \Leftrightarrow x_v = 1
  \end{aligned}
\end{equation}
i.e. the vertex \( v \in V \) belongs to the maximum independent set if and only if source entity with
a highest matching score are projected to the target candidate \( \tgt{p_v} \) that correspond to this vertex. In the case
when several source entities has equal maximum score we choose only the one entity.
In the instance of the MaxWIS problem nodes \( v, u \in V, u != v \) are connected if and only
if their corresponding target candidates are overlapping:
\[
  \{ v, u \} \in E \Leftrightarrow t_u \cap t_v \neq \emptyset
\].

Since we have a one to one correspondence between vertices from the MaxWIS problem and target
candidates from the set \( T \), consider the following set that fully determined by the solution of the
MaxWIS problem:
\[
  T^*_{x} = \{ t \in T | \exists s \in S, x_{s, t} = 1 \}
\]
% Let's \( x_1, x_2 \) be a two feasible solutions of problem \eqref{eq:ilp_without_nproj}, then we will say that
% they are in relation~\( \sim \) if their corresponding solutions of the MaxWIS problem are equal:
% \[
%   x_1 \sim x_2 \Leftrightarrow T^{*}_{x_1} = T^{*}_{x_2}
% \]
% Since this relation defined by an equality of sets it is reflexive, symmetric and transitive and therefore
% equivalence relation.

% Then quotient set \( \quot{X}{\sim} \) will consist of all equivalence classes of feasible solutions
% that differ only in matching with source entities. Every such equivalence class corresponds to one feasible solution of
% the maximum weight independent set problem.

Let's note that all feasible solutions of the MaxWIS will correspond to feasible solutions of the
problem \ref{eq:ilp_without_nproj} and vice versa.

\begin{lemma} \label{lemma:maxis_f_implies_ilp}
  Suppose \( x_v \) is a feasible solution of the maximum weight independent set problem, then there is a
  corresponding feasible solution of the problem \eqref{eq:ilp_without_nproj}.
\end{lemma}
\begin{proof}
  Suppose some non feasible solution \( x \) that corresponds to the feasible solution of the
  MaxWIS problem. Then we have:
  \[
    \exists (\src{p_1}, \src{p_2}, \tgt{p_1}, \tgt{p_2}) \in \hat{\Pi}(S, T) \Big|
    x_{\src{p_1}, \tgt{p_1}} + x_{\src{p_2}, \tgt{p_2}} > 1
  \]
  But since the solution \( x_v \) is feasible constraints with  \( \tgt{p_1} \neq \tgt{p_2} \) can not be violated:
  \begin{align*}
    & \forall \{ u, w \} \in E \quad x_u + x_w \leq 1 \stackrel{\eqref{eq:overlap_reduction}}{\implies}                  \\
    & \forall \tgt{p_1} \in T^*_x \; \nexists \tgt{p_2} \in T^*_x \Big| \tgt{p_1} \cap \tgt{p_2} \neq \emptyset \implies \\
    & \forall \src{p_1}, \src{p_2} \in S \quad x_{\src{p_1}, \tgt{p_1}} + x_{\src{p_2}, \tgt{p_2}} \leq 1
  \end{align*}
  Therefore \( \tgt{p_1} = \tgt{p_2} \) and constraints are violated because there exists
  at least two source nodes \( \src{p_1}, \src{p_2} \in S \) that are projected onto the same target nodes \( \tgt{p} \in T \):
  \[
    x_{\src{p_1}, \tgt{p}} = 1 \qquad x_{\src{p_2}, \tgt{p}} = 1
  \]
  But it contradict the fact that by construction of the corresponding solution of the ILP problem \eqref{eq:link_variables}
  there is only one source entity that is projected on every target candidates.
\end{proof}

\begin{lemma} \label{lemma:ilp_f_implies_maxis}
  Suppose \( x \) is a feasible solution of the problem \eqref{eq:ilp_without_nproj}, then
  the corresponding solution of the maximum weight independent set problem is feasible as well.
\end{lemma}
\begin{proof} By construction the non-overlapping constraints of the problem \eqref{eq:ilp_without_nproj}
  imply constraints of the MaxWIS problem:
  \begin{align*}
    & \forall (\src{p_1}, \src{p_2}, \tgt{p_1}, \tgt{p_2}) \in \hat{\Pi}(S, T) \quad
    x_{\src{p_1}, \tgt{p_1}} + x_{\src{p_2}, \tgt{p_2}} \leq 1 \implies                                                     \\
    & \forall \tgt{p_1}, \tgt{p_2} \in T, \tgt{p_1} \neq \tgt{p_2}, \tgt{p_1} \cap \tgt{p_2} \neq \emptyset
    \quad \nexists \src{p_1}, \src{p_2} \in S \Big|
    x_{\src{p_1}, \tgt{p_1}} + x_{\src{p_2}, \tgt{p_2}} > 1 \implies                                                     \\
    & \nexists \tgt{p_1}, \tgt{p_2} \in T, \tgt{p_1} \neq \tgt{p_2}, \tgt{p_1} \cap \tgt{p_2} \neq \emptyset \Big|
    \tgt{p_1} \in T^*_x, \tgt{p_2} \in T^*_x \implies                                                                    \\
    & \forall \{ v_{\tgt{p_1}}, v_{\tgt{p_2}} \} \in E \quad x_{v_{\tgt{p_1}}} + x_{v_{\tgt{p_2}}}\leq 1
  \end{align*}
\end{proof}

And finally we can show that it is possible to reduce the projection ILP problem without constraints \eqref{eq:num_proj_const}
to an instance of maximum weight independent set problem and therefore it can be solved by a polynomial-time algorithm.
\begin{theorem}
  The form \eqref{eq:ilp_without_nproj} of the projection ILP problem without constraints on number of target candidates projected from
  every source entity can be solved in a polynomial-time.
\end{theorem}
\begin{proof}

\end{proof}

Thus, The question whether the problem \eqref{eq:ilp} is NP-hard still remains open, but if it is,
constraints \eqref{eq:num_proj_const} are responsible for that.

\subsection{Approaches to compute the solution of the problem}
As any ILP problem the problem \eqref{eq:ilp} can be solved by branch and bound, cutting planes,
branch and cut methods, other exact algorithms. Nevertheless it can take a lot of time to find an optimal solution and
it can make such formulation inefficient from the application point of view where it is required to
solve instances of this problem for thousands sentences in a limited time.

It motivates the necessity of the approximate algorithm that can compute a solution that is not always
optimal, but don't violate crucial constraints. One of the approach for such an algorithm for the problem
\eqref{eq:ilp} is to iteratively assign \( 1 \) to a variable with the highest matching score and then remove
all target candidates that overlap with the projected candidate to enforce non-overlapping constraints.
Algorithm \ref{alg:ilp_greedy} is the variant of a such greedy algorithm.

\begin{algorithm}
  \caption{Approximate greedy algorithm for the proposed ILP problem} \label{alg:ilp_greedy}
  \KwData{instance of the ILP problem \eqref{eq:ilp}}
  \KwResult{\( x \) -- "solution" of the ILP problem}

  \( x \gets 0 \) \;
  \( P \gets 0 \) \Comment*[r]{number of projections by source entity}
  \While{\( \exists \src{p} \in S, \tgt{p} \in T \Big| c_{\src{p}, \tgt{p}} > 0 \)}{
    \( s, t \gets \argmax\limits_{\src{p} \in S, \tgt{p} \in T } c_{\src{p}, \tgt{p}} \) \;
    \( x_{s,t} \gets 1 \) \;
    \( P_{s} \gets P_{s} + 1 \) \;

    \ForAll(\tcp*[f]{remove all overlapping with \( t \) candidates}){\( \hat{t} \in T \Big| \hat{t} \cap t \neq \emptyset \)}{
      \( c_{s, \hat{t}} \gets 0 \) \;
    }

    \Comment{try to ensure constraints \eqref{eq:num_proj_const}}
    \If{constraints \eqref{eq:num_proj_const} is a type of \( =, \leq \)}{
      \If{\( P_s = n_{proj} \)}{
        \ForAll{\( \hat{t} \in T \)}{
          \( c_{s, \hat{t}} \gets 0 \) \;
        }
      }
    }
    \If{constraints \eqref{eq:num_proj_const} is a type of \( < \)}
    {
      \If{\( P_s = n_{proj} - 1 \)}{
        \ForAll{\( \hat{t} \in T \)}{
          \( c_{s, \hat{t}} \gets 0 \) \;
        }
      }
    }
  }
\end{algorithm}

From the steps of the algorithm \ref{alg:ilp_greedy} it implies that after removing all
overlapping target candidates by making their costs equal to zero the algorithm ensures the
constants \eqref{eq:num_proj_const} on a number of projected from every source entity candidates.
But only for the case where constraints have a form of \( <, \leq \).

In the case of \( =, >, \geq \) the algorithm will output a "solution" that violates constraint
\eqref{eq:num_proj_const} only in two cases. The first one is when there is a source entity for which
all target candidates, that are not overlapping with already projected ones, initially have zero matching cost. But then,
from the application point of view it doesn't make sense to project the source entity onto these candidates
since it is definitely not a counterpart of the source entity in the target sentence, otherwise it would not
have a matching cost equal to \( 0 \). The second option is the situation where target candidates
that could potentially be projected onto have been removed on the previous iterations of the algorithm.
And it is hard to handle this issue without backtracking and losing feasibility on
non-overlapping constraints \eqref{eq:non_overlap_const}.

Nevertheless, the algorithm \ref{alg:ilp_greedy} is linear on a number of variables and therefore can
solve the ILP problem significantly faster than the exact ILP solver in a general case.

Even in the case when the output of the greedy algorithm is a feasible solution it does not
necessary mean that the solution is optimal, i.e. there may exists a feasible solution with higher objective
value. This facts leads from the fact that the proposed ILP problem is NP-hard and therefore polynomial
algorithm can not solve it in a general case unless P=NP. But also it can be easily shown on an example.
Consider the ILP problem \eqref{eq:ilp} with \( n_{proj} = 2 \) and \( \leq \) type of constraints \eqref{eq:num_proj_const}.
Let \( T = \{ \tgt{p_1} = (1, 2), \tgt{p_2} = (2, 3), \tgt{p_3} = (3, 5) \}, S = \{ \src{p} \} \) and
the matching scores are the following:
\[
  c_{\src{p}, \tgt{p_1}} = 0.2 \qquad
  c_{\src{p}, \tgt{p_2}} = 0.3 \qquad
  c_{\src{p}, \tgt{p_3}} = 0.2
\]
The output of the algorithm \ref{alg:ilp_greedy} is the solution \( \src{p} \) with only one
non zero variable where \( x_{\src, \tgt{p_2}} = 1 \). It has an objective value \( 0.3 \).
While the optimal solution is the one where source entity is projected onto \( \tgt{p_1} \) and
\( \tgt{p_2} \) with the objective value equal to \( 0.4 \).
