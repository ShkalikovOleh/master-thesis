\usepackage{tudscrsupervisor} % script for creatiing the task description
\usepackage{tudscrcolor}

\usepackage[utf8]{inputenc}
\usepackage[T1]{fontenc}

\usepackage{babel}
\usepackage{csquotes}

\usepackage{isodate}
\usepackage{blindtext}

\usepackage{setspace}
\usepackage{acronym}
\usepackage{scrhack} 	% acronyms result in warning without this
\usepackage{multicol} 	% use multiple columns, used for the acronyms section

\usepackage{enumitem}\setlist{noitemsep} % used for the bullet points in the task section
\usepackage{microtype}	% better spacing

\usepackage{amsmath}
\usepackage{amsfonts}
\usepackage{amsthm}
\usepackage{amssymb}
\usepackage{bm}			% used for making things bold in equations
\usepackage[b]{esvect}

\usepackage{algorithm2e}
\RestyleAlgo{ruled}

\usepackage{multirow}	% use tables with columns stretching over multiple rows
\usepackage{booktabs}
\setlength\heavyrulewidth{0.25ex}
\usepackage{longtable}

\usepackage[font=footnotesize, format=plain, labelfont=bf]{caption}  %
\usepackage{subcaption}	% Packages to allow subfigures

\usepackage[bottom, hang, flushmargin]{footmisc}
\renewcommand\hangfootparindent{1em}
%\usepackage{fnpct}

% use modern bib package
\usepackage[
	backend=biber,
	style=alphabetic,
	sorting=ynt]{biblatex}
\addbibresource{../references.bib}

%\usepackage{xcolor}
\usepackage{listings}	% Package for displaying code
\definecolor{KeywordBlue}{cmyk}{0.88,0.77,0,0} %88,77,0,0
\definecolor{CommentGreen}{cmyk}{0.87,0.24,1.0,0.13} %87,24,100,13
\lstset{basicstyle=\scriptsize\ttfamily, language=C, commentstyle=\color{CommentGreen}, keywordstyle=\ttfamily\color{KeywordBlue}, backgroundcolor =\color[rgb]{0.95,0.95,0.95}, breaklines=true,literate={\\\%}{{\textcolor{black}{\\\%}}}1}


% some of the metadata for the pdf are defined in the title-file,
% as there are variables like author and title, whích would appear twice otherwise
% hyperref should always be the last package to be loaded
\usepackage[
	colorlinks=true,
	urlcolor=.,
	citecolor=.,
	linkcolor=.,
	pdfstartview=FitV,
	pdfdisplaydoctitle=true,
	hyperfootnotes=false
]{hyperref}
\urlstyle{same}		% use the same font for URLs as for the text

%%% PARAMS %%%
\pdfminorversion=7	% creates pdfs in the version 1.7, which prevents a warning with the logo

% Allow for triple digit page numbers in the toc
\makeatletter
\renewcommand*\@pnumwidth{2.1em}
\renewcommand*\@tocrmarg{3.1em}
\makeatother

\KOMAoptions{toc=chapterentrydotfill} 	% Add dots in toc for chapters
\setstretch{1.1}						% Adds a bit of space between the lines
\frenchspacing							% Only a single space after a dot


% Parameters to reduce 'Orphans' and 'Widdows'
\clubpenalty 			= 9999
\widowpenalty 			= 9999
\displaywidowpenalty   	= 1602
\brokenpenalty			= 4999	% Parameter for word disjuction on a pagebreak
\pretolerance			= 1100	% Parameter for difference from choosen format
\tolerance 				= 100 	% Parameter for difference from choosen format

% Less coservative parameters for floating objects in LaTeX
% An overview can be found in the book
% The Latex Companions Chapter 6.1
% A good start is
% http://robjhyndman.com/researchtips/latex-floats/

\setcounter{topnumber}{2}
\setcounter{bottomnumber}{2}
\setcounter{totalnumber}{4}
\renewcommand{\topfraction}{0.85}
\renewcommand{\bottomfraction}{0.85}
\renewcommand{\textfraction}{0.15}
\renewcommand{\floatpagefraction}{0.7}
\renewcommand{\textfraction}{0.1}
\setlength{\floatsep}{5pt plus 2pt minus 2pt}
%\setlength{\textfloatsep}{15pt plus 2pt minus 2pt}
%\setlength{\intextsep}{5pt plus 2pt minus 2pt}

\newtheorem{theorem}{Theorem}[section]
\newtheorem{corollary}{Corollary}[theorem]
\newtheorem{lemma}[theorem]{Lemma}
\theoremstyle{definition}
\newtheorem{definition}{Definition}[section]
\theoremstyle{remark}
\newtheorem*{remark}{Remark}

\newcommand{\src}[1]{#1^{src}}
\newcommand{\tgt}[1]{#1^{tgt}}
\DeclareMathOperator*{\argmax}{arg\,max}