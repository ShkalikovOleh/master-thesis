\chapter{Appendix II}
\label{sec:gen_ilp_is_np_hard}

Consider the generalized ILP problem \eqref{eq:objective}--\eqref{eq:binary_prog} where
the relation \( \cap \) is an arbitrary non-transitive, symmetric, and reflexive relation.
In a general case, it makes the initial ILP problem harder, i.e. it can be shown that the problem
\eqref{eq:objective}--\eqref{eq:binary_prog} become an NP-hard.

The usual way to prove the complexity of some problem is to reduce other problems with known complexity to
an instance of the studied one. Thus let's consider the maximum independent set problem \cite{pemmaraju2003computational}.

\begin{figure}[ht]
  \begin{subfigure}{.5\textwidth}
    \centering
    \begin{tikzpicture}[every node/.style = {draw, circle}]
      \node[fill=green!10] (1) {1};
      \node[right=of 1] (2) {2};
      \node[below right=of 2, fill=green!10] (3) {3};
      \node[below=of 1, fill=green!10] (4) {4};
      \node[left=of 4] (5) {5};
      \node[below right=of 4] (6) {6};

      \graph{
        (1) -- (2) -- (4) -- (6) -- (5) -- (4),
        (6) -- (2) -- (3)
      };
    \end{tikzpicture}
    \caption{Maximum independent set problem (MaxIS). Vertices that form an optimal solution are colored in \textbf{\textcolor{green!50}{green}}}
  \end{subfigure}
  \begin{subfigure}{.5\textwidth}
    \centering
    \begin{tikzpicture}[
        node distance=0.1,
        realstate/.style = {draw, circle, font=\scriptsize},
        dummystate/.style = {draw, circle, minimum width=4, inner sep=6, fill=blue!10}
      ]
      \node[dummystate] (src_1) {};
      \node[right=of src_1, xshift=10, dummystate] (src_2) {};
      \node[right=of src_2, xshift=10, dummystate] (src_3) {};

      \node[below=of src_1, fill=green!10, xshift=-50, yshift=-1.75cm, realstate] (1) {1};
      \node[right=of 1, realstate] (2) {2};
      \node[right=of 2, fill=green!10, realstate] (3) {3};
      \node[right=of 3, fill=green!10, realstate] (4) {4};
      \node[right=of 4, realstate] (5) {5};
      \node[right=of 5, realstate] (6) {6};
      \node[right=of 6, dummystate](dummy_1) {};
      \node[right=of dummy_1] (dots) {\dots};
      \node[right=of dots, dummystate] (dummy_2) {};

      \node[above=of src_2, text=gray, font=\scriptsize] {Source nodes};
      \node[below=of 5, text=gray, font=\scriptsize] {Target nodes};

      \draw[green] (src_1) -- (1);
      \draw (src_1) -- node[above left, xshift=3, rotate=60, black, font=\footnotesize]{c=1} (2);
      \draw (src_1) -- (3);
      \draw (src_1) -- (4);
      \draw (src_1) -- (5);
      \draw (src_1) -- (6);
      \draw[gray!40] (src_1) -- (dummy_1);
      \draw[gray!40] (src_1) -- (dummy_2);
      \draw (src_2) -- (2);
      \draw[green] (src_2) -- (3);
      \draw (src_2) -- (4);
      \draw (src_2) -- (5);
      \draw (src_2) -- (6);
      \draw[gray!40] (src_2) -- (dummy_1);
      \draw[gray!40] (src_2) -- (dummy_2);
      \draw (src_3) -- (2);
      \draw (src_3) -- (3);
      \draw[green] (src_3) -- (4);
      \draw (src_3) -- (5);
      \draw (src_3) -- (6);
      \draw[gray!40] (src_3) -- (dummy_1);
      \draw[gray!40] (src_3) -- node[above right, black, font=\footnotesize]{c=0} (dummy_2);
    \end{tikzpicture}
    \caption{Reduction to the generalized ILP problem. Dummy nodes are colored in \textcolor{blue!60}{blue}}
  \end{subfigure}
  \caption{The diagram of the general idea of the reduction MaxIS problem to the generalized ILP problem}
  \label{fig:maxis_reduction}
\end{figure}

The maximum independent set problem is a problem of finding a subset of vertices of some undirected
graph with maximum cardinality, such that it doesn't contain any vertices connected by edges of the graph.
An example of the maximum independent set problem is depicted in Figure \ref{fig:maxis_reduction} (a).
The formal definition of an ILP problem is given by \ref{def:maxis}.
\begin{definition}[MaxIS] \label{def:maxis}
  Let \( G=(V, E), V \neq \emptyset \) be an undirected graph, then a maximum independent set problem for the graph \( G \) is the following:
  \begin{align*}
    & \max \sum\limits_{v \in V} x_v                               \\
    & x_u + x_v \leq 1 \qquad \forall \{u, v\} \in E \\
    & x_v \in \{0, 1\}
  \end{align*}
\end{definition}

The MaxIS problem is an NP-hard \cite{pemmaraju2003computational} problem since VertexCover can be reduced to an instance of
this problem. But even more interesting for us is that this problem looks very similar to
our projection ILP problem, except that set of constraint \eqref{eq:num_proj_const} is
omitted and costs are equal to \( 1 \). Moreover, whether two edges of the graph are adjacent
is not transitive as well as the relation of overlapping \ref{def:overlapping}! That's why we will try to reduce
the maximum independent set problem to an instance of the generalized ILP problem.

The idea of the reduction is straightforward. For every vertex \( v \in V \) of the MaxIS problem
we will create one distinct target candidate \( \tgt{p_v} \) that corresponds to this vertex.
The set of all such target candidates will be called \( T_V \). The ILP problem is formulated as a projection
from source entities, to represent MaxIS as this ILP problem let's create dummy source nodes. The number of source nodes
is determined so that it will be easy to satisfy constraints~\eqref{eq:num_proj_const}.
The generalized number of dummy source nodes for any type of constraint \eqref{eq:num_proj_const} are given
in the table \ref{tab:dummy_nodes_num} (a). The matching cost between any dummy source node and
target candidate from the set \( T_V \) will be equal to \( 1 \). And finally, let's link the solution of the
ILP problem to the solutions of the MaxIS problem:
\[
  x_v = 1 \Leftrightarrow \exists \src{p} \in S \Big| x_{\src{p}, \tgt{p_v}} = 1
\]
i.e. the vertex \( v \in V \) belongs to the maximum independent set if and only if there is a dummy
source node that are projected to the target candidate \( \tgt{p_v} \) that correspond to this vertex.

\begin{table}[h]
  \begin{subtable}[t]{0.5\linewidth}
    \centering
    \renewcommand{\arraystretch}{1.5}
    \begin{tabular}{|c|c|c|}
      \hline
      \(<\)                                 & \( \leq \), \( = \)               & \(>\), \( \geq \) \\
      \hline
      \( \ceil{\frac{|V|}{n_{proj} - 1}} \) & \( \ceil{\frac{|V|}{n_{proj}}} \) & 1                 \\
      \hline
    \end{tabular}
    \caption{Source nodes}
    \label{tab:src_dummy_nodes_num}
  \end{subtable}
  \begin{subtable}[t]{0.5\linewidth}
    \centering
    \renewcommand{\arraystretch}{1.5}
    \begin{tabular}{|c|c|c|c|}
      \hline
      \(<\), \(\leq\) & \( = \)                                    & \(>\)          & \( \geq \)         \\
      \hline
      \( 0 \)         & \( \ceil{\frac{|V|}{n_{proj}}} n_{proj} \) & \( n_{proj} \) & \( n_{proj} + 1 \) \\
      \hline
    \end{tabular}
    \caption{Target nodes}
    \label{tab:tgt_dummy_nodes_num}
  \end{subtable}
  \caption{Number of dummy nodes for every type of constraints \eqref{eq:num_proj_const}}
  \label{tab:dummy_nodes_num}
\end{table}

We will consider that target nodes \( t_u, t_v \in T_{V} \) that correspond to graph's nodes \( u, v \in V \) overlap
if and only if there is an edge in the graph \( G \) between these nodes or \( u = v \).
\begin{equation} \label{eq:overlap_reduction}
  t_u \cap t_v \neq \emptyset \Leftrightarrow
  \left( \{ u, v \} \in E \right) \lor  \left(u = v \right)
\end{equation}
Such a definition of overlapping allows us to match all properties, i.e. reflexivity, symmetricity, and nontransitivity,
of overlapping relation on word ranges in the initial definition \ref{def:overlapping}. The fact that we consider
overlapping nodes that correspond to the same vertex simply means that it is impossible to
add the same vertex in the independent set two times which is a property of any set.

Hence the resulting ILP problem induced by the MaxIS problem is the following:
\begin{equation} \label{eq:reduction_without_nproj}
  \begin{aligned}
    & \max\limits_x \sum\limits_{(\src{p}, \tgt{p}) \in S \times T} c_{\src{p}, \tgt{p}} x_{\src{p}, \tgt{p}}                                             \\
    & \text{subject to}                                                                                                                                   \\
    & x_{\src{p_1}, \tgt{p_1}} + x_{\src{p_2}, \tgt{p_2}} \leq 1
    & \forall (\src{p_1}, \src{p_2}, \tgt{p_1}, \tgt{p_2}) \in \hat{\Pi}(S, T)                                                                            \\
    & x_{\src{p}, \tgt{p}} \in \{ 0, 1 \}                                                                     & \forall (\src{p}, \tgt{p}) \in S \times T
  \end{aligned}
\end{equation}
where set \( \hat{\Pi} \) defined based on the introduced above relation \( \cap \).

The problem \eqref{eq:reduction_without_nproj} is exactly the same as the ILP problem \eqref{eq:objective}--\eqref{eq:binary_prog}
except the fact that constraints \eqref{eq:num_proj_const} are omitted. The reason is that we can not be sure
that they can be satisfied for the type of constraints with \( =, >, \geq \) inequalities since it is possible to just run out of nodes.
To overcome this problem, let's create dummy target nodes. The generalized minimum number of such nodes is given
in the table \ref{tab:dummy_nodes_num} (b). Let's denote the set of all such nodes as \( T_{dummy} \) and then the set of all target candidates
will be \( T = T_V \cup T_{dummy} \). We will consider that dummy target nodes don't overlap with any node except itself:
\[
  \forall \tgt{p_1} \in T_{dummy}, \; \tgt{p} \in T \quad
  \tgt{p_1} \cap \tgt{p_2} \neq \emptyset \Leftrightarrow \tgt{p_1} = \tgt{p_2}
\]
Target dummy nodes are created only to satisfy constraints \eqref{eq:num_proj_const}, so all matching scores
between any dummy source node and any dummy target node set to be equal \( 0 \).

Since we have a one-to-one correspondence between vertices from the MaxIS problem and the target
candidates from the set \( T \), consider the following set that fully determines the solution of the
MaxIS problem:
\[
  T^*_{x} = \{ t \in T | \exists s \in S, x_{s, t} = 1 \}
\]
Let's \( x_1, x_2 \) be a two feasible solutions of problem \eqref{eq:ilp_without_nproj}, then we will say that
they are in relation~\( \sim \) if their corresponding solutions of the MaxIS problem are equal:
\[
  x_1 \sim x_2 \Leftrightarrow T^{*}_{x_1} = T^{*}_{x_2}
\]
Since this relation is defined by equality of sets, it is reflexive, symmetric, and transitive and therefore
equivalence relation.

Then quotient set \( \quot{X}{\sim} \) will consist of all equivalence classes of feasible solutions
that differ only in matching with dummy nodes. Every such equivalence class corresponds to one feasible solution of
the maximum weight-independent set problem.

Let's notice that the objective function value is equal for all
elements within any equivalence class.
\begin{lemma}
  Let X be a set of feasible solutions of the problem \eqref{eq:reduction_without_nproj}. Then
  for any equivalence class in the quotient set \( \quot{X}{\sim} \) the objective
  value is the same for every solution within this equivalence class.
\end{lemma}
\begin{proof}
  Consider an equivalence class \( [x] \) of some feasible solution \( x \).
  Let's notice that non-zero scores have only elements from the set \( T^*_x \),
  therefore the objective determines by this set of projected target candidates:
  \begin{multline*}
    \sum\limits_{(\src{p}, \tgt{p}) \in S \times T} c_{\src{p}, \tgt{p}} x_{\src{p}, \tgt{p}} =
    \sum\limits_{(\src{p}, \tgt{p}) \in S \times T_V} c_{\src{p}, \tgt{p}} x_{\src{p}, \tgt{p}} = \\
    \sum\limits_{(\src{p}, \tgt{p}) \in S \times T^{*}_{x}} c_{\src{p}, \tgt{p}} x_{\src{p}, \tgt{p}}
  \end{multline*}
  Because of non-overlapping constraints and the fact that we consider that the target candidate overlaps with
  itself only one source entity can be projected onto every target candidate from the set \( T_x^* \):
  \[
    \forall \tgt{p} \in T^*_x, \; \exists! \src{p} \in S \Big| x_{\src{p}, \tgt{p}} = 1
  \]
  Then the objective value equals the cardinality of the set \( T^*_x \) that are equal for
  any solution without the same equivalence class by the definition of the relation \( ~ \).
  \[
    \sum\limits_{(\src{p}, \tgt{p}) \in S \times T} c_{\src{p}, \tgt{p}} x_{\src{p}, \tgt{p}} =
    \sum\limits_{(\src{p}, \tgt{p}) \in S \times T^{*}_{x}} c_{\src{p}, \tgt{p}} x_{\src{p}, \tgt{p}} =
    |T_x^*|
  \]
\end{proof}
\begin{corollary} \label{col:maxis_and_ilp_objective_equal}
  The objective function value of the ILP problem \eqref{eq:reduction_without_nproj} and the maximum independent
  set problems are equal.
\end{corollary}
\begin{proof}
  By construction, every target candidate from the set \( T^*_x \) corresponds to only one vertex
  from the MaxIS problem. Therefore the objective value is equal to the cardinality of the set \( T^*_x \).
  In the proof of the lemma, we showed that the objective value of the ILP problem~\eqref{eq:reduction_without_nproj}
  equals to \( | T^*_x| \) as well.
\end{proof}

Dummy source nodes exist only to be able to formulate the MaxIS problem as the desired ILP projection problem.
Hence we can swap a dummy source node for a target candidate it was projected from.
\begin{lemma} \label{lemma:swap_source_reduction}
  Let \( x \) be a feasible solution of the problem \eqref{eq:reduction_without_nproj} such that
  some dummy source node \( \src{p} \in S \) is projected onto \( \tgt{p} \in T \),
  i.e. \( x_{\src{p}, \tgt{p}} = 1 \). Then for any other source node
  \( \src{\hat{p}} \in S, \src{p} \neq \src{\hat{p}} \) a solution \( x^{*} \), such that
  \begin{align*}
    & x^*_{\src{\hat{p}}, \tgt{p}} = 1 \qquad
    x^*_{\src{p}, \tgt{p}} = 0                                                                                            \\
    & \forall (s, t) \in S \times T \setminus \{ (\src{p}, \tgt{p}), (\src{\hat{p}}, \tgt{p}) \} \quad x^*_{s,t} = x_{s,t}
  \end{align*}
  is also feasible and in the same equivalence class as \( x \).
\end{lemma}
\begin{proof}
  Suppose that the solution \( x^* \) is not feasible, it means that:
  \[
    \exists s \in S, t \in T \Big| (s, \src{\hat{p}}, t, \tgt{p}) \in \hat{\Pi}(S, T)
    \quad x^*_{s, t} = 1
  \]
  Let's notice that \( s \neq \src{p} \), since \( x^*_{\src{p}, \tgt{p}} = 0 \).
  But then by construction:
  \[
    x_{\src{p}, \tgt{p}} + x_{s, t} = 1 + x^*_{s, t} = 1 + 1 > 2
  \]
  It contradicts with the fact that \( x \) is a feasible solution, therefore \( x^* \) should be
  feasible.

  And since by construction sets \( T^*_x \) and \( T^*_{x^*} \) are equal solutions are in the same equivalence
  class and by the corollary \ref{col:maxis_and_ilp_objective_equal} the objective values for these solutions are also equal.
\end{proof}

\begin{corollary} \label{col:bound_num_proj_reduction}
  For any feasible solution of the problem \eqref{eq:reduction_without_nproj} there is a feasible solution  \( x^* \)
  such that
  \[
    \sum\limits_{\tgt{p} \in T} x^*_{\src{p}, \tgt{p}} \leq \frac{|V|}{|S|}
    \qquad \forall \src{p} \in S
  \]
  and \( T_{x^*}^* =  T^*_x \).
\end{corollary}
\begin{proof}
  Assume the solution \( \hat{x} \) such that
  \begin{align*}
    & \forall \src{p} \in S \;\; \forall \tgt{p} \in T_V \qquad \hat{x}_{\src{p}, \tgt{p}} = x_{\src{p}, \tgt{p}} \\
    & \forall \src{p} \in S \;\; \forall \tgt{p} \in T_{dummy} \qquad \hat{x}_{\src{p}, \tgt{p}} = 0
  \end{align*}
  By construction, this solution has the same objective value since all scores for dummy target nodes are equal to \( 0 \),
  the same set \( T^*_{\hat{x}} = T^*_x \)
  and also it doesn't violate the non-overlapping constraints:
  \begin{align*}
    & \forall (\src{p_1}, \src{p_2}, \tgt{p_1}, \tgt{p_2}) \in \hat{\Pi}(S, T) \Big| \tgt{p_2} \in T_{V} \quad
    \hat{x}_{\src{p_1}, \tgt{p_1}} + \hat{x}_{\src{p_2}, \tgt{p_2}} =
    x_{\src{p_1}, \tgt{p_1}} + x_{\src{p_2}, \tgt{p_2}} \leq 1                                                      \\
    & \forall (\src{p_1}, \src{p_2}, \tgt{p_1}, \tgt{p_2}) \in \hat{\Pi}(S, T) \Big| \tgt{p_2} \in T_{dummy} \quad
    \hat{x}_{\src{p_1}, \tgt{p_1}} + 0 \leq
    x_{\src{p_1}, \tgt{p_1}} + x_{\src{p_2}, \tgt{p_2}} \leq 1
  \end{align*}

  Because of non-overlapping constraints \eqref{eq:non_overlap_const} at most one source node can
  be projected on every target node, therefore:
  \[
    \sum\limits_{\src{p} \in S} \sum\limits_{\tgt{p} \in T} \hat{x}_{\src{p}, \tgt{p}} \leq |T_V| \leq |V|
  \]
  Suppose there exists such a \( \src{p} \in S \) that
  \[
    \sum\limits_{\tgt{p} \in T} \hat{x}_{\src{p}, \tgt{p}} > \frac{|V|}{|S|}
  \]
  Then there is another source entity \( \src{\hat{p}} \) for which we have
  \[
    \sum\limits_{\tgt{p} \in T} \hat{x}_{\src{\hat{p}}, \tgt{p}} < \frac{|V|}{|S|}
  \]
  By the lemma \ref{lemma:swap_source_reduction} there exists such an optimal solution \( x^* \)
  where
  \begin{align*}
    & \sum\limits_{\tgt{p} \in T} x^*_{\src{p}, \tgt{p}} = \sum\limits_{\tgt{p} \in T} \hat{x}_{\src{p}, \tgt{p}} - 1             \\
    & \sum\limits_{\tgt{p} \in T} x^*_{\src{\hat{p}}, \tgt{p}} = 1 + \sum\limits_{\tgt{p} \in T} \hat{x}_{\src{\hat{p}}, \tgt{p}} \\
  \end{align*}
  and \( T^*_x = T^*_{\hat{x}} = T^*_{x^*}  \). If this solution doesn't satisfy the desired property repeat the procedure.
\end{proof}

\begin{lemma} \label{lemma:app_maxis_f_implies_ilp}
  Suppose \( x_v \) is a feasible solution of the maximum weight independent set problem, then there is a
  corresponding feasible solution of the problem \eqref{eq:reduction_without_nproj}.
\end{lemma}
\begin{proof}
  Suppose some non-feasible solution \( x \) that corresponds to the feasible solution of the
  MaxWIS problem. Then we have:
  \[
    \exists (\src{p_1}, \src{p_2}, \tgt{p_1}, \tgt{p_2}) \in \hat{\Pi}(S, T) \Big|
    x_{\src{p_1}, \tgt{p_1}} + x_{\src{p_2}, \tgt{p_2}} > 1
  \]
  By construction of the set \( \hat{\Pi}(S, T) \) and the overlapping relation (dummy target nodes
  are overlapping only with itself) if \( \tgt{p_1} \neq \tgt{p_2} \) then:
  \begin{multline*}
    \left\{ (\src{p_1}, \src{p_2}, \tgt{p_1}, \tgt{p_2}) \Big| (\src{p_1}, \src{p_2}, \tgt{p_1}, \tgt{p_2}) \in \hat{\Pi}(S, T) \right\} = \\
    \left\{ (\src{p_1}, \src{p_2}, \tgt{p_1}, \tgt{p_2}) \Big| (\src{p_1}, \src{p_2}, \tgt{p_1}, \tgt{p_2}) \in \hat{\Pi}(S, T_V) \right\}
  \end{multline*}
  But since the solution \( x_v \) is feasible constraints with  \( \tgt{p_1} \neq \tgt{p_2} \) can not be violated:
  \begin{align*}
    & \forall \{ u, w \} \in E \quad x_u + x_w \leq 1 \stackrel{\eqref{eq:overlap_reduction}}{\implies}                  \\
    & \forall \tgt{p_1} \in T^*_x \; \nexists \tgt{p_2} \in T^*_x \Big| \tgt{p_1} \cap \tgt{p_2} \neq \emptyset \implies \\
    & \forall \src{p_1}, \src{p_2} \in S \quad x_{\src{p_1}, \tgt{p_1}} + x_{\src{p_2}, \tgt{p_2}} \leq 1
  \end{align*}
  Therefore \( \tgt{p_1} = \tgt{p_2} \) and constraints are violated because there exists
  at least two source nodes \( \src{p_1}, \src{p_2} \in S \) that are projected onto the same target nodes \( \tgt{p} \in T \):
  \[
    x_{\src{p_1}, \tgt{p}} = 1 \qquad x_{\src{p_2}, \tgt{p}} = 1
  \]

  But then consider a solution \( x^* \) such that we remove one of the projections that
  make this constraint violated:
  \begin{align*}
    & \forall (s,t) \in S \times T \setminus \left\{ (\src{p_2}, \tgt{p}) \right\}
    \quad x^*_{s,t} = x_{s,t} \\
    & x^*_{\src{p_2}, \tgt{p}} = 0
  \end{align*}
  Note that \( T^*_{x^*} = T^*_x  \) because there is a projection to the target candidate \( \tgt{p} \)
  since \( \)  \( x^*_{\src{p_1}, \tgt{p}} = 1 \).
  If the solution \( x^* \) is infeasible - repeat the procedure, otherwise
  this solution is feasible and corresponds to the same solution \( x_v \) of the MaxWIS problem.
\end{proof}

And finally, we can prove the complexity of the generalized ILP problem by reducing the MaxIS problem
to an instance of \eqref{eq:objective}--\eqref{eq:binary_prog}.
\begin{theorem}[\( MaxIS \leq_P ILP_{proj} \)]
  The generalized ILP problem \eqref{eq:objective}--\eqref{eq:binary_prog} is NP-hard
\end{theorem}
\begin{proof}
  Assume the reduction of the MaxIS problem to an instance of the projection ILP problem described above.
  Since the number of nodes in the ILP problem is linear on the number of vertices in the MaxIS problem and
  has a solution to the ILP problem it scales at most quadratically on a number of vertices time to
  determine the solution of the MaxIS problem the reduction takes a polynomial time.
  The only thing that is required to check is whether the projected optimal solution to the ILP problem
  will be always the optimal solution of the MaxIS and vice versa.

  By the corollary \ref{col:maxis_and_ilp_objective_equal} the objective values of two problems are equal and therefore
  feasible solution with the highest objective value will have the highest objective value in the counterpart problem.
  Then by lemmas \ref{lemma:app_maxis_f_implies_ilp} and \ref{lemma:ilp_f_implies_maxis} the non-overlapping constraint
  won't make any optimal solution infeasible for their counterpart.

  But besides non-overlapping constraint the problem \eqref{eq:objective}--\eqref{eq:binary_prog} also has
  constraints \eqref{eq:num_proj_const}. Let's check whether they won't make any optimal solution of the MaxIS
  problem infeasible in their ILP formulation and if there are no new optimal solutions.

  Consider a set \( X \) of all feasible solutions of the problem \eqref{eq:reduction_without_nproj} and
  set \( Y \) of all feasible solutions of the problem \eqref{eq:objective}--\eqref{eq:binary_prog}. Since
  the latter problem is constrained version of the problem \eqref{eq:reduction_without_nproj} \( Y \subset X \), but then:
  \[
    Y \subset X \implies \forall y \in Y \quad \exists x \in X \Big| \quad [y] \subset [x],
  \]
  i.e. there won't be any new equivalence classes and therefore no new feasible solutions of the MaxIS problem.
  But it turns out that for any equivalence class there is a feasible solution of the full generalized problem that belongs to this class:
  \[
    \forall x \in X \quad \exists y \in Y \Big| \quad y \in [x]
  \]
  Let's prove it. For this, we need to analyze constraints \eqref{eq:num_proj_const} for every type of inequality.

  \textit{The case of \( < \).} The number of dummy source nodes is given in the table \ref{tab:dummy_nodes_num} (a) and equal to
  \( |S| = \ceil{\frac{|V|}{n_{proj} - 1}} \)
  By the corollary \ref{col:bound_num_proj_reduction} there exists such a solution \( x^* \) from the same
  equivalence class that satisfies constraints:
  \[
    \forall \src{p} \in S \quad
    \sum\limits_{\tgt{p} \in T} x^*_{\src{p}, \tgt{p}} \leq \frac{|V|}{|S|} =
    \frac{|V|}{ \ceil{\frac{|V|}{n_{proj} - 1}}} \leq \frac{|V|}{\frac{|V|}{n_{proj} - 1}} =
    n_{proj} - 1 < n_{proj}.
  \]

  \textit{The case of \( \leq \).} By the same corollary \ref{col:bound_num_proj_reduction} as in the previous case
  we obtain that there is a solution \( x^* \) that satisfies constraints:
  \[
    \forall \src{p} \in S \quad
    \sum\limits_{\tgt{p} \in T} x^*_{\src{p}, \tgt{p}} \leq \frac{|V|}{|S|} =
    \frac{|V|}{ \ceil{\frac{|V|}{n_{proj}}}} \leq \frac{|V|}{\frac{|V|}{n_{proj}}} =
    n_{proj} .
  \]

  \textit{The case of \( =\).} By the corollary \ref{col:bound_num_proj_reduction}
  for any solution \( x \) from the equivalence class \( [x] \) there is
  a feasible solution \( x^* \in [x] \) such that:
  \begin{equation} \label{eq:red_thm_n_proj}
    \forall \src{p} \in S \quad
    \sum\limits_{\tgt{p} \in T} x^*_{\src{p}, \tgt{p}} \leq \frac{|V|}{|S|} =
    \frac{|V|}{ \ceil{\frac{|V|}{n_{proj}}}} \leq \frac{|V|}{\frac{|V|}{n_{proj}}} =
    n_{proj} .
  \end{equation}
  If constraints \eqref{eq:num_proj_const} are violated then there is such a source node \( \src{\hat{p}} \)
  for which the sum is a strong inequality:
  \begin{equation} \label{eq:red_thm_src_that_violate_constr}
    \exists \src{\hat{p}} \in S \quad
    \sum\limits_{\tgt{p} \in T} x^*_{\src{p}, \tgt{p}} \leq \frac{|V|}{|S|} <
    n_{proj} .
  \end{equation}
  But then we can find such a dummy target node \( t_d \in T_{dummy} \) that:
  \[
    \exists t_d \in T_{dummy} \Big| \forall s \in S \quad x^*_{s,t} = 0
  \]
  It can be proved by contradiction to the property of the given by the corollary \ref{col:bound_num_proj_reduction}
  and that there exists such a \( \src{\hat{p}} \):
  \begin{align*}
    & \nexists t_d \in T_{dummy} \Big| \forall s \in S \quad x^*_{s,t} = 0 \implies                                                             \\
    & \forall t \in T_{dummy}, \; \exists s \in S \quad x^*_{s,t} = 1 \implies                                                                  \\
    & \ceil{\frac{|V|}{n_{proj}}} n_{proj} = |S| n_{proj}
    \stackrel{\eqref{eq:red_thm_n_proj}}{\geq}
    \sum\limits_{s \in S} \sum\limits_{t \in T} x^*_{s,t} \geq |T_{dummy}| = \ceil{\frac{|V|}{n_{proj}}} n_{proj} \implies                       \\
    & \sum\limits_{s \in S} \sum\limits_{t \in T} x^*_{s,t} = \ceil{\frac{|V|}{n_{proj}}} n_{proj} \geq |V|
    \implies                                                                                                                                     \\
    & \sum\limits_{s \in S \setminus \{ \src{\hat{p}} \}} \sum\limits_{t \in T} x^*_{s,t} = \ceil{\frac{|V|}{n_{proj}}} n_{proj} = |S| n_{proj}
    \stackrel{\eqref{eq:red_thm_src_that_violate_constr}}{\implies}                                                                              \\
    & \exists s \in S \quad
    \sum\limits_{\tgt{p} \in T} x^*_{\src{p}, \tgt{p}} > \frac{|V|}{|S|} >
    n_{proj}
  \end{align*}
  In short words always there are a "free" dummy target nodes that can be used to ensure equality constraints.
  It was expected just because during construction we chose the number of target dummy nodes to make it possible.
  To make the solution closer to being feasible we just need to project source entity \( \src{\hat{p}} \) onto this target
  candidate \( t_d \). So, the modified solution \( \hat{x} \) is the following:
  \begin{align*}
    & \forall (s,t) \in S \times T \setminus \left\{ (\src{\hat{p}}, t_d) \right\} \hat{x}_{s,t} = x^*_{s,t} \\
    & \hat{x}_{\src{\hat{p}}, t_d} = 1
  \end{align*}
  By the definition of the set \( T^*_{\hat{x}} \) since we add only a projection to a dummy node
  it won't change the equivalence class of the solution and objective value, but make the number of projection
  from the source entity \( \src{\hat{p}} \) higher:
  \[
    \sum\limits_{\tgt{p} \in T} \hat{x}_{\src{\hat{p}}, \tgt{p}} =
    \sum\limits_{\tgt{p} \in T} x^*_{\src{\hat{p}}, \tgt{p}} + 1
    \leq n_{proj}
  \]
  Since dummy target nodes overlap only with itself we have the constraints that potentially
  can be violated are defined by this set:
  \begin{multline*}
    \left\{ (\src{p_1}, \src{p_2}, t_d, \tgt{p_2}) \Big| (\src{p_1}, \src{p_2}, \tgt{p_1}, \tgt{p_2}) \in \hat{\Pi}(S, T) \right\} = \\
    \left\{ (\src{p_1}, \src{p_2}, t_d, t_d) \Big| \src{p_1}, \src{p_2} \in S, \src{p_1} \neq \src{p_2} \right\}
  \end{multline*}
  But because of the choice of \( t_d \) the constraints
  \eqref{eq:non_overlap_const} are not violated:
  \begin{align*}
    \forall \src{p_1}, \src{p_2} \in S \Big| \src{p_1} \neq \src{p_2}, \src{p_1} \neq \src{\hat{p}}, \src{p_2} \neq \src{\hat{p}} \quad
    & \hat{x}_{src{p_1}, t_d} + \hat{x}_{\src{p_2}, t_d} = x_{\src{p_1}, t_d} + x_{\src{p_2}, t_d} = 0 \\
    \forall \src{p_2} \in S \Big| \src{\hat{p}} \neq \src{p_2} \quad
    & \hat{x}_{\src{\hat{p}}, t_d} + \hat{x}_{\src{p_2}, t_d} = 1 + 0 = 1 \leq 1
  \end{align*}

  If still there are some source entities for which constrains \eqref{eq:num_proj_const} are violated - repeat the process.
  Repetition of the process is possible since the property \eqref{eq:red_thm_n_proj} still holds and
  we proved that there always exists a "free" dummy target node we can project this source node onto.

  \textit{The case of \( >, \geq \).} We can perform a similar derivation as for the case of \( = \) type of constraints, but
  let's note that \( \forall \src{p} \in S \):
  \begin{align*}
    & \sum\limits_{\tgt{p} \in T} x_{\src{p}, \tgt{p}} = n_{proj} \Leftrightarrow
    n_{proj} \geq \sum\limits_{\tgt{p} \in T} x_{\src{p}, \tgt{p}} \geq n_{proj}                                  \\
    & \bigvee\limits_{k=n_{proj} + 1}^{|T|} \sum\limits_{\tgt{p} \in T} x_{\src{p}, \tgt{p}} = k \Leftrightarrow
    \sum\limits_{\tgt{p} \in T} x^*_{\src{p}, \tgt{p}} > n_{proj} \qquad
  \end{align*}
  Therefore all solutions that are feasible for the case of equality constraints are also should be
  feasible for the case of \( \geq, > \) and vice versa, so we won't lose any equivalence class.

  So, we have shown that for every feasible solution of the problem \eqref{eq:reduction_without_nproj} there is
  a feasible solution of the full generalized ILP problem \eqref{eq:objective}--\eqref{eq:binary_prog} that belongs to the
  same equivalence class. And every feasible solution of the full generalized ILP problem is a feasible solution of the problem \eqref{eq:reduction_without_nproj}.
  Therefore the MaxIS problem can be solved as an instance of the problem \eqref{eq:objective}--\eqref{eq:binary_prog}
  and consequently, the generalized ILP problem is at least as hard as the maximum independent set problem which is NP-hard.
\end{proof}
